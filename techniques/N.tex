%%%%(c)
%%%%(c)  This file is a portion of the source for the textbook
%%%%(c)
%%%%(c)    A First Course in Linear Algebra
%%%%(c)    Copyright 2004 by Robert A. Beezer
%%%%(c)
%%%%(c)  See the file COPYING.txt for copying conditions
%%%%(c)
%%%%(c)
\begin{para}When we construct the contrapositive of a theorem (\acronymref{technique}{CP}), we need to negate the two statements in the implication.  And when we construct a proof by contradiction (\acronymref{technique}{CD}), we need to negate the conclusion of the theorem.  One way to construct a converse (\acronymref{technique}{CV}) is to simultaneously negate the hypothesis and conclusion of an implication (but remember that this is not guaranteed to be a true statement).  So we often have the need to negate statements, and in some situations it can be tricky.\end{para}
%
\begin{para}If a statement says that a set is empty, then its negation is the statement that the set is nonempty.  That's straightforward.  Suppose a statement says ``something-happens'' for all $i$, or every $i$, or any $i$.  Then the negation is that  ``something-doesn't-happen'' for at least one value of $i$.  If a statement says that there exists at least one ``thing,'' then the negation is the statement that there is no ``thing.''  If a statement says that a ``thing'' is unique, then the negation is that there is zero, or more than one, of the ``thing.''\end{para}
%
\begin{para}We are not covering all of the possibilities, but we wish to make the point that logical qualifiers like ``there exists'' or ``for every'' must be handled with care when negating statements.  Studying the proofs which employ contradiction (as listed in \acronymref{technique}{CD}) is a good first step towards understanding the range of possibilities.\end{para}