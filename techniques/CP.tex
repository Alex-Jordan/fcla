%%%%(c)
%%%%(c)  This file is a portion of the source for the textbook
%%%%(c)
%%%%(c)    A First Course in Linear Algebra
%%%%(c)    Copyright 2004 by Robert A. Beezer
%%%%(c)
%%%%(c)  See the file COPYING.txt for copying conditions
%%%%(c)
%%%%(c)
The \define{contrapositive} of an implication $P\Rightarrow Q$ is the implication ${\rm not}(Q)\Rightarrow{\rm not}(P)$, where ``not'' means the logical negation, or opposite.  An implication is true if and only if its contrapositive is true.  In symbols, $(P\Rightarrow Q)\iff({\rm not}(Q)\Rightarrow{\rm not}(P))$ is a theorem.  Such statements about logic, that are always true, are known as \define{tautologies}.\par
%
For example, it is a theorem that ``if a vehicle is a fire truck, then it has big tires and has a siren.''  (Yes, I'm sure you can conjure up a counterexample, but play along with me anyway.)  The contrapositive is  ``if a vehicle does not have big tires or does not have a siren, then it is not a fire truck.''  Notice how the ``and'' became an ``or'' when we negated the conclusion of the original theorem.\par
%
It will frequently happen that it is easier to construct a proof of the contrapositive than of the original implication.  If you are having difficulty formulating a proof of some implication, see if the contrapositive is easier for you.  The trick is to construct the negation of complicated statements accurately.  More on that later.
