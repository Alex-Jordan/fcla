%%%%(c)
%%%%(c)  This file is a portion of the source for the textbook
%%%%(c)
%%%%(c)    A First Course in Linear Algebra
%%%%(c)    Copyright 2004 by Robert A. Beezer
%%%%(c)
%%%%(c)  See the file COPYING.txt for copying conditions
%%%%(c)
%%%%(c)
\begin{para}A definition is a made-up term, used as a kind of shortcut for some typically more complicated idea.  For example, we say a whole number is \define{even} as a shortcut for saying that when we divide the number by two we get a remainder of zero.  With a precise definition, we can answer certain questions unambiguously.  For example, did you ever wonder if zero was an even number?  Now the answer should be clear since we have a precise definition of what we mean by the term even.\end{para}
%
\begin{para}A single term might have several possible definitions.  For example, we could say that the whole number $n$ is even if there is another whole number $k$ such that $n=2k$.  We say this is an equivalent definition since it categorizes even numbers the same way our first definition does.\end{para}
%
\begin{para}Definitions are like two-way streets --- we can use a definition to replace something rather complicated by its definition (if it fits) {\em and} we can replace a definition by its more complicated description.  A definition is usually written as some form of an implication, such as ``If something-nice-happens, then {\bf blatzo}.''  However, this also means that ``If blatzo, then something-nice-happens,'' even though this may not be formally stated.  This is what we mean when we say a definition is a two-way street --- it is really two implications, going in opposite ``directions.''\end{para}
%
\begin{para}Anybody (including you) can make up a definition, so long as it is unambiguous, but the real test of a definition's utility is whether or not it is useful for describing interesting or frequent situations.\end{para}
%
\begin{para}We will talk about theorems later (and especially equivalences).  For now, be sure not to confuse the notion of a definition with that of a theorem.\end{para}
%
\begin{para}In this book, we will display every new definition carefully set-off from the text, and the term being defined will be written thus: \define{definition}.  Additionally, there is a full list of all the definitions, in order of their appearance located in the reference section of the same name (\miscref{definition}{Definitions}).  Definitions are critical to doing mathematics and proving theorems, so we've given you lots of ways to locate a definition should you forget its\dots uh, well, \dots definition.\end{para}
%
\begin{para}Can you formulate a precise definition for what it means for a number to be \define{odd}?  (Don't just say it is the opposite of even.  Act as if you don't have a definition for even yet.)  Can you formulate your definition a second, equivalent, way?  Can you employ your definition to test an odd and an even number for ``odd-ness''?\end{para}
