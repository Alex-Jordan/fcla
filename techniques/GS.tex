%%%%(c)
%%%%(c)  This file is a portion of the source for the textbook
%%%%(c)
%%%%(c)    A First Course in Linear Algebra
%%%%(c)    Copyright 2004 by Robert A. Beezer
%%%%(c)
%%%%(c)  See the file COPYING.txt for copying conditions
%%%%(c)
%%%%(c)
\begin{para}``I don't know how to get started!'' is often the lament of the novice proof-builder.  Here are a few pieces of advice.
%
\begin{enumerate}
%
\item  As mentioned in \acronymref{technique}{T}, rewrite the statement of the theorem in an ``if-then'' form.  This will simplify identifying the hypothesis and conclusion, which are referenced in the next few items.
%
\item  Ask yourself what {\em kind} of statement you are trying to prove.  This is always part of your conclusion.  Are you being asked to conclude that two numbers are equal, that a function is differentiable or a set is a subset of another?  You cannot bring other techniques to bear if you do not know what {\em type} of conclusion you have.
%
\item  Write down reformulations of your hypotheses.  Interpret and translate each definition properly.
%
\item Write your hypothesis at the top of a sheet of paper and your conclusion at the bottom.  See if you can formulate a statement that precedes the conclusion and also implies it.  Work down from your hypothesis, and up from your conclusion, and see if you can meet in the middle.  When you are finished, rewrite the proof nicely, from hypothesis to conclusion, with verifiable implications giving each subsequent statement.
%
\item As you work through your proof, think about what kinds of objects your symbols represent.  For example, suppose $A$ is a set and $f(x)$ is a real-valued function.  Then the expression $A+f$ might make no sense if we have not defined what it means to ``add'' a set to a function, so we can stop at that point and adjust accordingly.  On the other hand we might understand $2f$ to be the function whose rule is described by $(2f)(x)=2f(x)$.  ``Think about your objects'' means to always verify that your objects and operations are compatible.
%
\end{enumerate}
\end{para}
