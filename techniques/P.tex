%%%%(c)
%%%%(c)  This file is a portion of the source for the textbook
%%%%(c)
%%%%(c)    A First Course in Linear Algebra
%%%%(c)    Copyright 2004 by Robert A. Beezer
%%%%(c)
%%%%(c)  See the file COPYING.txt for copying conditions
%%%%(c)
%%%%(c)
\begin{para}Here is a technique used by many practicing mathematicians when they are teaching themselves new mathematics.  As they read a textbook, monograph or research article, they attempt to prove each new theorem themselves, {\em before} reading the proof.  Often the proofs can be very difficult, so it is wise not to spend too much time on each.  Maybe limit your losses and try each proof for 10 or 15 minutes.  Even if the proof is not found, it is time well-spent.  You become more familiar with the definitions involved, and the hypothesis and conclusion of the theorem.  When you do work through the proof, it might make more sense, and you will gain added insight about just how to construct a proof.\end{para}

