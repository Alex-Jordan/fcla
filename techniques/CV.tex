%%%%(c)
%%%%(c)  This file is a portion of the source for the textbook
%%%%(c)
%%%%(c)    A First Course in Linear Algebra
%%%%(c)    Copyright 2004 by Robert A. Beezer
%%%%(c)
%%%%(c)  See the file COPYING.txt for copying conditions
%%%%(c)
%%%%(c)
The \define{converse} of the implication $P\Rightarrow Q$ is the implication $Q\Rightarrow P$.  There is no guarantee that the truth of these two statements are related.  In particular, if an implication has been proven to be a theorem, then do not try to use its converse too, as if it were a theorem.  Sometimes the converse is true (and we have an equivalence, see \acronymref{technique}{E}).  But more likely the converse is false, especially if it wasn't included in the statement of the original theorem.\par
%
For example, we have the theorem, ``if a vehicle is a fire truck, then it is has big tires and has a siren.''  The converse is false.  The statement that ``if a vehicle has big tires and a siren, then it is a fire truck'' is false.  A police vehicle for use on a sandy public beach would have big tires and a siren, yet is not equipped to fight fires.\par
%
We bring this up now, because \acronymref{theorem}{CSRN} has a tempting converse.  Does this theorem say that if $r<n$, then the system is consistent?  Definitely not, as \acronymref{archetype}{E} has $r=3<4=n$, yet is inconsistent.  This example is then said to be a \define{counterexample} to the converse.  Whenever you think a theorem that is an implication might actually be an equivalence, it is good to hunt around for a counterexample that shows the converse to be false (the archetypes, \acronymref{appendix}{A}, can be a good hunting ground).
