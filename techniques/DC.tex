%%%%(c)
%%%%(c)  This file is a portion of the source for the textbook
%%%%(c)
%%%%(c)    A First Course in Linear Algebra
%%%%(c)    Copyright 2004 by Robert A. Beezer
%%%%(c)
%%%%(c)  See the file COPYING.txt for copying conditions
%%%%(c)
%%%%(c)
Much of your mathematical upbringing, especially once you began a study of algebra, revolved around simplifying expressions --- combining like terms, obtaining common denominators so as to add fractions, factoring in order to solve polynomial equations.  However, as often as not, we will do the opposite.  Many theorems and techniques will revolve around taking some object and ``decomposing'' it into some combination of other objects, ostensibly in a more complicated fashion.  When we say something can ``be written as'' something else, we mean that the one object can be decomposed into some combination of other objects.  This may seem unnatural at first, but results of this type will give us insight into the structure of the original object by exposing its inner workings.  An appropriate analogy might be stripping the wallboards away from the interior of a building to expose the structural members supporting the whole building.\par
%
Perhaps you have studied integral calculus, or a pre-calculus course, where you learned about partial fractions.  This is a technique where a fraction of two polynomials is {\em decomposed} (written as, expressed as) a sum of simpler fractions.  The purpose in calculus is to make finding an antiderivative simpler.  For example, you can verify the truth of the expression
%
\begin{align*}
\frac{{12 {x}^{5} } + {2 {x}^{4} } - {20 {x}^{3} } + {66 {x}^{2} } -{294 x} + 308}
{{x}^{6}  + {x}^{5}  - {3 {x}^{4} } + {21 {x}^{3} } - {52{x}^{2} } + {20 x} - 48}
&=
\frac{{5 x} + 2}{{x}^{2} - x + 6} +
\frac{{3 x} - 7}{{x}^{2}  + 1} +
\frac{3}{x + 4} +
\frac{1}{x - 2}
\end{align*}
%
In an early course in algebra, you might be expected to combine the four terms on the right over a common denominator to create the ``simpler'' expression on the left.  Going the other way, the partial fraction technique would allow you to systematically {\em decompose} the fraction of polynomials on the left into the sum of the four (arguably) simpler fractions of polynomials on the right.\par
%
This is a major shift in thinking, so come back here often, especially when we say ``can be written as'', or ``can be expressed as,'' or ``can be decomposed as.''
