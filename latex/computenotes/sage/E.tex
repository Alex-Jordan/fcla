%%%%(c)
%%%%(c)  This file is a portion of the source for the textbook
%%%%(c)
%%%%(c)    A First Course in Linear Algebra
%%%%(c)    Copyright 2004 by Robert A. Beezer
%%%%(c)
%%%%(c)  See the file COPYING.txt for copying conditions
%%%%(c)
%%%%(c)
\contributedby{\stevecanfield}\\
SAGE can compute eigenspaces and eigenvalues for you. If you have a matrix named \computerfont{a} and you type
\begin{equation*}
\computerfont{a.eigenspaces()}
\end{equation*}
you will get a listing of the eigenvalues and the eigenspace for each. Let's do an example.  Your output may be formatted slightly different from what we have here.
%
\begin{align*}
&\computerfont{m = matrix( QQ, [[-13,-8,-4], [12,7,4], [24,16,7]])} \\
&\computerfont{m.eigenspaces()} \\
&\computerfont{[(3, [(1, 2/3, 1/3)]),(-1, [(1, 0, 1/2),(0, 1, -1/2)])]}\\
\end{align*}
%
Whew, that looks like a mess. At the top level, eigenspaces() returns a dictionary whose keys are the eigenvalues. So in this case we have eigenvalues 3 and -1. Each eigenvalue has an array after it that forms the basis of the eigenspace. In our example, there is 1 vector for $\lambda=3$ and 2 vectors for $\lambda=-1$. Finally, the vectors SAGE spits out may not be the nicest ones to work with. In particular, we might want to scale the vectors to get rid of fractions.