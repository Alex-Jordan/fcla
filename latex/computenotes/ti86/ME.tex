%%%%(c)
%%%%(c)  This file is a portion of the source for the textbook
%%%%(c)
%%%%(c)    A First Course in Linear Algebra
%%%%(c)    Copyright 2004 by Robert A. Beezer
%%%%(c)
%%%%(c)  See the file COPYING.txt for copying conditions
%%%%(c)
%%%%(c)
On the TI-86, press the \computerfont{MATRX} key \computerfont{(Yellow-7)}.  Press the second menu key over, \computerfont{F2}, to bring up the \computerfont{EDIT} screen.  Give your matrix a name, one letter or many, then press \computerfont{ENTER}.  You can then change the size of the matrix (rows, then columns) and begin editing individual entries (which are initially zero).  \computerfont{ENTER} will move you from entry to entry, or the \computerfont{down arrow} key will move you to the next row.  A menu gives you extra options for editing.\par
%
Matrices may also be entered on the home screen as follows.  Use brackets ([~,~]) to enclose rows with elements separated by commas.  Group rows, in order, into a final set of brackets (with no commas between rows).  This can then be stored in a name with the \computerfont{STO} key.  So, for example,
%
\begin{equation*}
\computerfont{
[[1,2,3,4]\ [5,6,7,8]\ [9,10,11,12]]\rightarrow A
}
\end{equation*}
%
will create a matrix named \computerfont{A} that is equal to
%
\begin{equation*}
%
\begin{bmatrix}
1&2&3&4\\
5&6&7&8\\
9&10&11&12
\end{bmatrix}
%
\end{equation*}