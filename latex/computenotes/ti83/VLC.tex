%%%%(c)
%%%%(c)  This file is a portion of the source for the textbook
%%%%(c)
%%%%(c)    A First Course in Linear Algebra
%%%%(c)    Copyright 2004 by Robert A. Beezer
%%%%(c)
%%%%(c)  See the file COPYING.txt for copying conditions
%%%%(c)
%%%%(c)
\contributedby{\dougphelps}\\
Entering a vector on the TI-83 is the same process as entering a matrix.  You press \computerfont{4 ENTER 3 ENTER} for a $4\times 3$ matrix.  Likewise, you press \computerfont{4 ENTER 1 ENTER} for a vector of size 4.  To multiply a vector by 8, press the number 8,  then press the \computerfont{MATRX} key, then scroll down to the letter you named your vector (A, B, C, etc) and press \computerfont{ENTER}.\par
%
To add vectors \computerfont{A} and \computerfont{B} for example, press the \computerfont{MATRX} key, then \computerfont{ENTER}.  Then press the \computerfont{+} key.  Then press the \computerfont{MATRX} key, then the down arrow  once, then \computerfont{ENTER}.  \computerfont{[A] + [B]} will appear on the screen. Press \computerfont{ENTER}.
