%%%%(c)
%%%%(c)  This file is a portion of the source for the textbook
%%%%(c)
%%%%(c)    A First Course in Linear Algebra
%%%%(c)    Copyright 2004 by Robert A. Beezer
%%%%(c)
%%%%(c)  See the file COPYING.txt for copying conditions
%%%%(c)
%%%%(c)
Given a matrix $A$, Mathematica will compute a set of column vectors whose span is the null space of the matrix with the \computerfont{NullSpace[\,]} command.  Perhaps not coincidentally, this set is exactly $\setparts{\vect{z}_j}{1\leq j\leq n-r}$.  However, Mathematica prefers to output the vectors in the opposite order than one we have chosen.  Here's a small example.\par
%
Begin with the $3\times 4$ matrix $A$, and its row-reduced version $B$,
%
\begin{align*}
A&=
\begin{bmatrix}
 1 & 2 & -1 & 0 \\
 3 & 4 & 1 & -2 \\
 -1 & 1 & -5 & 3
\end{bmatrix}
&
&\rref
&
B&=
\begin{bmatrix}
 \leading{1} & 0 & 3 & -2 \\
 0 & \leading{1} & -2 & 1 \\
 0 & 0 & 0 & 0
\end{bmatrix}
\end{align*}
%
We could extract entries from $B$ to build the vectors $\vect{z}_1$ and $\vect{z}_2$ according to \acronymref{theorem}{SSNS} and describe $\nsp{A}$ as a span of the set $\set{\vect{z}_1,\,\vect{z}_2}$.  Instead, if $\computerfont{a}$ has been set to $A$, then executing the command \computerfont{NullSpace[a]} yields the list of lists (column vectors),
%
\begin{align*}
\{\{2, -1, 0, 1\}, \{-3, 2, 1, 0\}\}
\end{align*}
%
Notice how our $\vect{z}_1$ is second in the list.  To ``correct'' this we can use a list-processing command from Mathematica, \computerfont{Reverse[\,]}, as follows,
%
\begin{align*}
\text{\computerfont{Reverse[NullSpace[a]]}}
\end{align*}
%
and receive the output in our preferred order.  Give it a try yourself.
