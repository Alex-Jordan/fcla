%%%%(c)
%%%%(c)  This file is a portion of the source for the textbook
%%%%(c)
%%%%(c)    A First Course in Linear Algebra
%%%%(c)    Copyright 2004 by Robert A. Beezer
%%%%(c)
%%%%(c)  See the file COPYING.txt for copying conditions
%%%%(c)
%%%%(c)
Matrices are input as lists of lists, since a list is a basic data structure in {\sl Mathematica}.  A matrix is a list of rows, with each row entered as a list.  {\sl Mathematica} uses braces (($\left\lbrace\right.$~,~$\left.\right\rbrace$)) to delimit lists.  So the input
\begin{equation*}
\computerfont{a = \{\{1,2,3,4\}, \{5,6,7,8\},\{9,10,11,12\}\}}
\end{equation*}
%
would create a $3\times 4$ matrix named \computerfont{a} that is equal to
%
\begin{equation*}
%
\begin{bmatrix}
1&2&3&4\\
5&6&7&8\\
9&10&11&12
\end{bmatrix}
%
\end{equation*}
%
To display a matrix named \computerfont{a} ``nicely'' in {\sl Mathematica}, type \computerfont{MatrixForm[a]}, and the output will be displayed with rows and columns.  If you just type \computerfont{a}, then you will get a list of lists, like how you input the matrix in the first place.