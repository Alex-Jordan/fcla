Another way of expressing \acronymref{theorem}{RCLS} is to say a system is consistent if and only if column $n+1$ is not a pivot column of $B$.  Sage has the matrix method \verb?.pivot()? to easily identify the pivot columns of a matrix.  Let's use \acronymref{archetype}{E} as an example.
%
\begin{sageexample}
sage: coeff = matrix(QQ, [[ 2, 1,  7, -7],
...                       [-3, 4, -5, -6],
...                       [ 1, 1,  4, -5]])
sage: const = vector(QQ, [2, 3, 2])
sage: aug = coeff.augment(const)
sage: aug.rref()
[ 1  0  3 -2  0]
[ 0  1  1 -3  0]
[ 0  0  0  0  1]
\end{sageexample}
%
\begin{sageexample}
sage: aug.pivots()
(0, 1, 4)
\end{sageexample}
%
We can \emph{look} at the reduced row-echelon form of the augmented matrix and see a leading one in the final column, so we know the system is inconsistent.  However, we could just as easily not form the reduced row-echelon form and just look at the list of pivot columns computed by \verb?aug.pivots()?.  Since \verb?aug? has 5 columns, the final column is numbered \verb?4?, which is present in the list of pivot columns, as we expect.\par
%
One feature of Sage is that we can easily extend its capabilities by defining new commands.  Here will create a function that checks if an augmented matrix represents a consistent system of equations.  The syntax is just a bit complicated.  \verb?lambda? is the word that indicates we are making a new function, the input is temporarily named \verb?A? (think \verb?A?ugmented), and the \emph{name} of the function is \verb?consistent?.  Everything following the colon will be evaluated and reported back as the output.
%
\begin{sageexample}
sage: consistent = lambda A: not(A.ncols()-1 in A.pivots())
\end{sageexample}
%
Execute this block above.  There will not be any output, but now the \verb?consistent? function will be defined and available.  Now give it a try (after making sure to have run the code above that defines \verb?aug?).  Note that the output of \verb?consistent()? will be either \verb?True? or \verb?False?.
%
\begin{sageexample}
sage: consistent(aug)
False
\end{sageexample}
%
The \verb?consistent()? command works by simply checking to see if the last column of \verb?A? is \emph{not} in the list of pivots.  We can now test many different augmented matrices, such as perhaps changing the vector of constants while keeping the coefficient matrix fixed.  Again, make sure you execute the code above that defines \verb?coeff? and \verb?const?.
%
\begin{sageexample}
sage: consistent(coeff.augment(const))
False
sage: w = vector(QQ, [3,1,2])
sage: consistent(coeff.augment(w))
True
sage: u = vector(QQ, [1,3,1])
sage: consistent(coeff.augment(u))
False
\end{sageexample}
%
Why do some vectors of constants lead to a consistent system with this coefficient matrix, while others do not?  This is a fundamental question, which we will come to understand in several different ways.
%
\begin{sageverbatim}
\end{sageverbatim}
%