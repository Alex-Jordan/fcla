We have deferred several mysteries and postponed some explanations of the terms Sage uses to describe certain objects.  This is because Sage is a comprehensive tool that you can use throughout your mathematical career, and Sage assumes you already know a lot of linear algebra.  Don't worry, you already know \emph{some} linear algebra and will very soon will know a \emph{whole lot more}.  And we know enough now that we can now solve one mystery.  How can we create \emph{all} of the solutions to a linear system of equations when the solution set is infinite?\par
%
\acronymref{theorem}{VFSLS} described elements of a solution set as a single vector $\vect{c}$, plus linear combinations of vectors $\vectorlist{u}{n-r}$.  The vectors $\vect{u}_j$ are identical to the vectors $\vect{z}_j$ in the description of the spanning set of a null space in \acronymref{theorem}{SSNS}, suitably adjusted from the setting of a general system and the relevant homogeneous system for a null space.  We can get the single vector $\vect{c}$ from the \verb?.solve_right()? method of a coefficient matrix, once supplied with the vector of constants.  The vectors $\set{\vectorlist{z}{n-r}}$ come from Sage in the basis returned by the \verb?.right_kernel(basis='pivot')? method applied to the coefficient matrix.  \acronymref{theorem}{PSPHS} amplifies and generalizes this discussion, making it clear that the choice of the particular particular solution $\vect{c}$ (Sage's choice, and your author's choice) is just a matter of convenience.\par
%
In our previous discussion, we used the system from \acronymref{example}{ISSI}.  We will use it one more time.
%
\begin{sageexample}
sage: coeff = matrix(QQ, [[ 1,  4,  0, -1,  0,   7, -9],
...                       [ 2,  8, -1,  3,  9, -13,  7],
...                       [ 0,  0,  2, -3, -4,  12, -8],
...                       [-1, -4,  2,  4,  8, -31, 37]])
sage: n = coeff.ncols()
sage: const = vector(QQ, [3, 9, 1, 4])
sage: c = coeff.solve_right(const)
sage: c
(4, 0, 2, 1, 0, 0, 0)
sage: N = coeff.right_kernel(basis='pivot').basis()
sage: z1 = N[0]
sage: z2 = N[1]
sage: z3 = N[2]
sage: z4 = N[3]
sage: soln = c + 2*z1 - 3*z2 - 5*z3 + 8*z4
sage: soln
(31, 2, -50, -71, -3, -5, 8)
sage: check = sum([soln[i]*coeff.column(i) for i in range(n)])
sage: check
(3, 9, 1, 4)
sage: check == const
True
\end{sageexample}
%
So we can describe the infinitely many solutions with just five items: the specific solution, $\vect{c}$, and the four vectors of the spanning set.  Boom!\par
%
As an exercise in understanding \acronymref{theorem}{PSPHS}, you might begin with \verb?soln? and add a linear combination of the spanning set for the null space to create another solution (which you can check).
%
\begin{sageverbatim}
\end{sageverbatim}
%