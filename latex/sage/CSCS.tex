We could compute the column space of a matrix with a span of the set of columns of the matrix, much as we did back in \acronymref{sage}{CSS} when we were checking consistency of linear systems using spans of the set of columns of a coeffiicent matrix.  However, Sage provides a convenient matrix method to construct this same span: \verb?.column_space()?.  Here is a check.
%
\begin{sageexample}
sage: D = matrix(QQ, [[ 2, -1, -4],
...                   [ 5,  2, -1],
...                   [-3,  1,  5]])
sage: cs = D.column_space(); cs
Vector space of degree 3 and dimension 2 over Rational Field
Basis matrix:
[    1     0 -11/9]
[    0     1  -1/9]
sage: cs_span = (QQ^3).span(D.columns())
sage: cs == cs_span
True
\end{sageexample}
%
In \acronymref{sage}{CSS} we discussed solutions to systems of equations and the membership of the vector of constants in the span of the columns of the coefficient matrix.  This discussion turned on \acronymref{theorem}{SLSLC}.  We can now be slightly more efficient with \acronymref{theorem}{CSCS}, while still using the same ideas.  We recycle the computations from \acronymref{example}{CSMCS} that use \acronymref{archetype}{D} and \acronymref{archetype}{E}.
%
\begin{sageexample}
sage: coeff = matrix(QQ, [[ 2, 1,  7, -7],
...                       [-3, 4, -5, -6],
...                       [ 1, 1,  4, -5]])
sage: constD = vector(QQ, [8, -12, 4])
sage: constE = vector(QQ, [2, 3, 2])
sage: cs = coeff.column_space()
sage: coeff.solve_right(constD)
(4, 0, 0, 0)
sage: constD in cs
True
sage: coeff.solve_right(constE)
Traceback (most recent call last):
...
ValueError: matrix equation has no solutions
sage: constE in cs
False
\end{sageexample}
%
\begin{sageverbatim}
\end{sageverbatim}
%








