Being nonsingular is an important matrix property, and in such cases Sage contains commands that quickly and easily determine if the mathematical object does, or does not, have the property.  The names of these types of methods universally begin with \verb?.is_?, and these might be referred to as ``predicates'' or ``queries.''.  In the Sage notebook, define a simple matrix \verb?A?, and then in a cell type \verb?A.is_?, followed by pressing the tab key rather than evaluating the cell.  You will get a list of numerous properties that you can investigate for the matrix \verb?A?.  (This will not work as advertised with the Sage cell server.)
%
The other convention is to name these properties in a positive way, so the relevant command for nonsingular matrices is \verb?.is_singular()?.  We will redo \acronymref{example}{S} and \acronymref{example}{NM}.  Note the use of \verb!not! in the last compute cell.
%
\begin{sageexample}
sage: A = matrix(QQ, [[1, -1, 2],
...                   [2,  1, 1],
...                   [1,  1, 0]])
sage: A.is_singular()
True
\end{sageexample}
%
\begin{sageexample}
sage: B = matrix(QQ, [[-7, -6, -12],
...                   [ 5,  5,   7],
...                   [ 1,  0,   4]])
sage: B.is_singular()
False
sage: not(B.is_singular())
True
\end{sageexample}
%
\begin{sageverbatim}
\end{sageverbatim}
%
