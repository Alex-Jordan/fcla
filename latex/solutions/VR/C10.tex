%%%%(c)
%%%%(c)  This file is a portion of the source for the textbook
%%%%(c)
%%%%(c)    A First Course in Linear Algebra
%%%%(c)    Copyright 2004 by Robert A. Beezer
%%%%(c)
%%%%(c)  See the file COPYING.txt for copying conditions
%%%%(c)
%%%%(c)
We need to express the vector $\vect{v}$ as a linear combination of the vectors in 
$B$.  \acronymref{theorem}{VRRB} tells us we will be able to do this, and do it uniquely.  The vector equation
%
\begin{equation*}
a_1\colvector{2\\-2\\2}+
a_2\colvector{1\\3\\1}+
a_3\colvector{3\\5\\2}
=
\colvector{11\\5\\8}
\end{equation*}
%
becomes (via \acronymref{theorem}{SLSLC}) a system of linear equations with augmented matrix,
%
\begin{equation*}
\begin{bmatrix}
2 & 1 & 3 & 11\\
-2 & 3 & 5 & 5\\
2 & 1 & 2 & 8
\end{bmatrix}
\end{equation*}
%
This system has the unique solution $a_1=2$, $a_2=-2$, $a_3=3$.  So by \acronymref{definition}{VR},
%
\begin{equation*}
\vectrep{B}{\vect{v}}=\vectrep{B}{\colvector{11\\5\\8}}
=
\vectrep{B}{
2\colvector{2\\-2\\2}+
(-2)\colvector{1\\3\\1}+
3\colvector{3\\5\\2}
}
=\colvector{2\\-2\\3}
\end{equation*}
