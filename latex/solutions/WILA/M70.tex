%%%%(c)
%%%%(c)  This file is a portion of the source for the textbook
%%%%(c)
%%%%(c)    A First Course in Linear Algebra
%%%%(c)    Copyright 2004 by Robert A. Beezer
%%%%(c)
%%%%(c)  See the file COPYING.txt for copying conditions
%%%%(c)
%%%%(c)
If the price of standard mix is set at \$5.292, then the profit function has a zero coefficient on the variable quantity $f$.  So, we can set $f$ to be any integer quantity in $\set{825,\,826,\,\ldots,\,960}$.  All but the extreme values ($f=825$, $f=960$) will result in production levels where some of every mix is manufactured.  No matter what value of $f$ is chosen, the resulting profit will be the same, at \$2,664.60.