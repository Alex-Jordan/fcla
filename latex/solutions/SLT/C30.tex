%%%%(c)
%%%%(c)  This file is a portion of the source for the textbook
%%%%(c)
%%%%(c)    A First Course in Linear Algebra
%%%%(c)    Copyright 2004 by Robert A. Beezer
%%%%(c)
%%%%(c)  See the file COPYING.txt for copying conditions
%%%%(c)
%%%%(c)
If we transform the basis of $P_4$, then \acronymref{theorem}{SSRLT} guarantees we will have a spanning set of $\rng{T}$.  A basis of $P_4$ is $\set{1, x, x^2, x^3, x^4}$.  If we transform the elements of this set, we get the set $\set{0, 1, 2x, 3x^2, 4x^3}$ which is a spanning set for $\rng{T}$.  Reducing this to a linearly independent set, we find that $\{1, 2x, 3x^2, 4x^3\}$ is a basis of $\rng{T}$.  Since $\rng{T}$ and $P_3$ both have dimension 4, $T$ is surjective.  