%%%%(c)
%%%%(c)  This file is a portion of the source for the textbook
%%%%(c)
%%%%(c)    A First Course in Linear Algebra
%%%%(c)    Copyright 2004 by Robert A. Beezer
%%%%(c)
%%%%(c)  See the file COPYING.txt for copying conditions
%%%%(c)
%%%%(c)
We wish to find an output vector $\vect{v}$ that has no associated input.  This is the same as requiring that there is no solution to the equality
%
\begin{equation*}
\vect{v}=\lt{T}{\colvector{a\\b\\c}}=\colvector{2a+3b-c\\2b-2c\\a-b+2c}
=a\colvector{2\\0\\1}+b\colvector{3\\2\\-1}+c\colvector{-1\\-2\\2}
\end{equation*}
%
In other words, we would like to find an element of $\complex{3}$ not in the set
%
\begin{equation*}
Y=\spn{\set{\colvector{2\\0\\1},\,\colvector{3\\2\\-1},\,\colvector{-1\\-2\\2}}}
\end{equation*}
%
If we make these vectors the rows of a matrix, and row-reduce, \acronymref{theorem}{BRS} provides an alternate description of $Y$,
%
\begin{equation*}
Y=\spn{\set{\colvector{2\\0\\1},\,\colvector{0\\4\\-5}}}
\end{equation*}
%
If we add these vectors together, and then change the third component of the result, we will create a vector that lies outside of $Y$, say $\vect{v}=\colvector{2\\4\\9}$.
