%%%%(c)
%%%%(c)  This file is a portion of the source for the textbook
%%%%(c)
%%%%(c)    A First Course in Linear Algebra
%%%%(c)    Copyright 2004 by Robert A. Beezer
%%%%(c)
%%%%(c)  See the file COPYING.txt for copying conditions
%%%%(c)
%%%%(c)
To find an element of $\complex{3}$ with an empty pre-image, we will compute the range of the linear transformation $\rng{S}$ and then find an element outside of this set.\par
%
By \acronymref{theorem}{SSRLT} we can evaluate $S$ with the elements of a spanning set of the domain and create a spanning set for the range.
%
\begin{align*}
\lt{S}{\colvector{1\\0\\0\\0}}&=\colvector{2\\1\\-1}
&
\lt{S}{\colvector{0\\1\\0\\0}}&=\colvector{1\\3\\2}
&
\lt{S}{\colvector{0\\0\\1\\0}}&=\colvector{3\\4\\1}
&
\lt{S}{\colvector{0\\0\\0\\1}}&=\colvector{-4\\3\\7}
\end{align*}
%
So
%
\begin{equation*}
\rng{S}=\spn{\set{
\colvector{2\\1\\-1},\,
\colvector{1\\3\\2},\,
\colvector{3\\4\\1},\,
\colvector{-4\\3\\7}
}}
\end{equation*}
%
This spanning set is obviously linearly dependent, so we can reduce it to a basis for $\rng{S}$ using \acronymref{theorem}{BRS}, where the elements of the spanning set are placed as the rows of a matrix.  The result is that
%
\begin{equation*}
\rng{S}=\spn{\set{
\colvector{1\\0\\-1},\,
\colvector{0\\1\\1}
}}
\end{equation*}
%
Therefore, the unique vector in $\rng{S}$ with a first slot equal to 6 and a second slot equal to 15 will be the linear combination
%
\begin{equation*}
6\colvector{1\\0\\-1}+15\colvector{0\\1\\1}=\colvector{6\\15\\9}
\end{equation*}
%
So, any vector with first two components equal to 6 and 15, but with a third component different from 9, such as
%
\begin{equation*}
\vect{w}=\colvector{6\\15\\-63}
\end{equation*}
%
will not be an element of the range of $S$ and will therefore have an empty pre-image.
%
Another strategy on this problem is to {\em guess}.  Almost any vector will lie outside the range of $T$, you have to be unlucky to randomly choose an element of the range.  This is because the codomain has dimension 3, while the range is ``much smaller'' at a dimension of 2.  You still need to check that your guess lies outside of the range, which generally will involve solving a system of equations that turns out to be inconsistent.