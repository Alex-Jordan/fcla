%%%%(c)
%%%%(c)  This file is a portion of the source for the textbook
%%%%(c)
%%%%(c)    A First Course in Linear Algebra
%%%%(c)    Copyright 2004 by Robert A. Beezer
%%%%(c)
%%%%(c)  See the file COPYING.txt for copying conditions
%%%%(c)
%%%%(c)
If we transform the basis of $P_2$, then \acronymref{theorem}{SSRLT} guarantees we will have a spanning set of $\rng{T}$.  A basis of $P_2$ is $\set{1, x, x^2}$.  If we transform the elements of this set, we get the set $\set{x^2, x^3, x^4}$ which is a spanning set for $\rng{T}$.  These three vectors are linearly independent, so $\set{x^2, x^3, x^4}$ is a basis of $\rng{T}$.