%%%%(c)
%%%%(c)  This file is a portion of the source for the textbook
%%%%(c)
%%%%(c)    A First Course in Linear Algebra
%%%%(c)    Copyright 2004 by Robert A. Beezer
%%%%(c)
%%%%(c)  See the file COPYING.txt for copying conditions
%%%%(c)
%%%%(c)
(a)\quad First find a matrix $B$ that is row-equivalent to $A$ and in reduced row-echelon form
%
\begin{equation*}
B=
\begin{bmatrix}
\leading{1} & 0 & 3 & -2\\ 
0 & \leading{1} & 1 & -1\\ 
0 & 0 & 0 & 0
\end{bmatrix}
\end{equation*}
%
By \acronymref{theorem}{BCS} we can choose the columns of $A$ that correspond to dependent variables ($D=\set{1,2}$) as the elements of $S$ and obtain the desired properties.  So
%
\begin{equation*}
S=\set{\colvector{2\\-5\\1},\,\colvector{-1\\3\\1}}
\end{equation*}
%
(b)\quad  We can write the column space of $A$ as the row space of the transpose (\acronymref{theorem}{CSRST}).  So we row-reduce the transpose of $A$ to obtain the row-equivalent matrix $C$ in reduced row-echelon form
%
\begin{equation*}
C=
\begin{bmatrix}
1 & 0 & 8\\ 
0 & 1 & 3\\ 
0 & 0 & 0\\ 
0 & 0 & 0
\end{bmatrix}
\end{equation*}
%
The nonzero rows (written as columns) will be a linearly independent set that spans the row space of $\transpose{A}$, by \acronymref{theorem}{BRS}, and the zeros and ones will be at the top of the vectors,
%
\begin{equation*}
S=\set{\colvector{1\\0\\8},\,\colvector{0\\1\\3}}
\end{equation*}
%
(c)\quad In preparation for \acronymref{theorem}{FS}, augment $A$ with the $3\times 3$ identity matrix $I_3$ and row-reduce to obtain the extended echelon form,
%
\begin{equation*}
\begin{bmatrix}
1 & 0 & 3 & -2 & 0 & -\frac{1}{8} & \frac{3}{8}\\
0 & 1 & 1 & -1 & 0 & \frac{1}{8} & \frac{5}{8}\\
0 & 0 & 0 & 0 & 1 & \frac{3}{8} & -\frac{1}{8}
\end{bmatrix}
\end{equation*}
%
Then since the first four columns of row 3 are all zeros, we extract
%
\begin{equation*}
L=
\begin{bmatrix}
\leading{1} & \frac{3}{8} & -\frac{1}{8}
\end{bmatrix}
\end{equation*}
%
\acronymref{theorem}{FS} says that $\csp{A}=\nsp{L}$.  We can then use  \acronymref{theorem}{BNS} to construct the desired set $S$, based on the free variables with indices in $F=\set{2,3}$ for the homogeneous system $\homosystem{L}$, so
%
\begin{equation*}
S=\set{\colvector{-\frac{3}{8}\\1\\0},\,\colvector{\frac{1}{8}\\0\\1}}
\end{equation*}
%
Notice that the zeros and ones are at the bottom of the vectors.\\
%
(d)\quad This is a straightforward application of \acronymref{theorem}{BRS}.  Use the row-reduced matrix $B$ from part (a), grab the nonzero rows, and write them as column vectors,
%
\begin{equation*}
S=\set{\colvector{1\\0\\3\\-2},\,\colvector{0\\1\\1\\-1}}
\end{equation*}
