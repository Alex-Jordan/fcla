%%%%(c)
%%%%(c)  This file is a portion of the source for the textbook
%%%%(c)
%%%%(c)    A First Course in Linear Algebra
%%%%(c)    Copyright 2004 by Robert A. Beezer
%%%%(c)
%%%%(c)  See the file COPYING.txt for copying conditions
%%%%(c)
%%%%(c)
If $B$ is singular, then there exists a vector $\vect{x}\ne\zerovector$ so that $\vect{x}\in \nsp{B}$.  Thus, $B\vect{x} = \vect{0}$, so $A(B\vect{x}) = (AB)\vect{x} = \zerovector$, so $\vect{x}\in\nsp{AB}$.  Since the null space of $AB$ is not trivial, $AB$ is a singular matrix.
