%%%%(c)
%%%%(c)  This file is a portion of the source for the textbook
%%%%(c)
%%%%(c)    A First Course in Linear Algebra
%%%%(c)    Copyright 2004 by Robert A. Beezer
%%%%(c)
%%%%(c)  See the file COPYING.txt for copying conditions
%%%%(c)
%%%%(c)
For the first pre-image, we want $\vect{x}\in\complex{3}$ such that $\lt{T}{\vect{x}}=\colvector{2\\3}$.  This becomes,
%
\begin{equation*}
\colvector{2x_1-x_2+5x_3\\-4x_1+2x_2-10x_3}=\colvector{2\\3}
\end{equation*}
%
Vector equality gives a system of two linear equations in three variables, represented by the augmented matrix
%
\begin{equation*}
\begin{bmatrix}
2 & -1 & 5 & 2\\
-4 & 2 & -10 & 3
\end{bmatrix}
\rref
\begin{bmatrix}
\leading{1} & -\frac{1}{2} & \frac{5}{2} & 0\\
0 & 0 & 0 & \leading{1}
\end{bmatrix}
\end{equation*}
%
so the system is inconsistent and the pre-image is the empty set.  For the second pre-image the same procedure leads to an augmented matrix with a different vector of constants
%
\begin{equation*}
\begin{bmatrix}
2 & -1 & 5 & 4\\
-4 & 2 & -10 & -8
\end{bmatrix}
\rref
\begin{bmatrix}
\leading{1} & -\frac{1}{2} & \frac{5}{2} & 2\\
0 & 0 & 0 & 0
\end{bmatrix}
\end{equation*}
%
This system is consistent and has infinitely many solutions, as we can see from the presence of the  two free variables ($x_2$ and $x_3$) both to zero.  We apply \acronymref{theorem}{VFSLS} to obtain
%
\begin{equation*}
\preimage{T}{\colvector{4\\-8}}=
\setparts{
\colvector{2\\0\\0}+
x_2\colvector{\frac{1}{2}\\1\\0}+
x_3\colvector{-\frac{5}{2}\\0\\1}
}{
x_2,\,x_3\in\complex{\null}
}
\end{equation*}
%
