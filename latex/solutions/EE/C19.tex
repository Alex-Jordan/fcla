%%%%(c)
%%%%(c)  This file is a portion of the source for the textbook
%%%%(c)
%%%%(c)    A First Course in Linear Algebra
%%%%(c)    Copyright 2004 by Robert A. Beezer
%%%%(c)
%%%%(c)  See the file COPYING.txt for copying conditions
%%%%(c)
%%%%(c)
First compute the characteristic polynomial,
%
\begin{align*}
\charpoly{C}{x}
&=\detname{C-xI_2}&&\text{\acronymref{definition}{CP}}\\
%
&=
\begin{vmatrix}
 -1-x & 2 \\
 -6    & 6-x
\end{vmatrix}\\
%
&=(-1-x)(6-x)-(2)(-6)\\
&=x^2-5x+6\\
&=(x-3)(x-2)
%
\end{align*}
%
So the eigenvalues of $C$ are the solutions to $\charpoly{C}{x}=0$, namely, $\lambda=2$ and $\lambda=3$.\par
%
To obtain the eigenspaces, construct the appropriate singular matrices and find expressions for the null spaces of these matrices.
\begin{align*}
%
\lambda&=2\\
C-(2)I_2&=
\begin{bmatrix}
-3 & 2\\
-6 & 4
\end{bmatrix}
\rref
\begin{bmatrix}
\leading{1} & -\frac{2}{3}\\
0 & 0
\end{bmatrix}\\
\eigenspace{C}{2}&=\nsp{C-(2)I_2}=
\spn{\set{\colvector{\frac{2}{3}\\1}}}
=
\spn{\set{\colvector{2\\3}}}
%
\end{align*}
%
\begin{align*}
%
\lambda&=3\\
C-(3)I_2&=
\begin{bmatrix}
-4 & 2\\
-6 & 3
\end{bmatrix}
\rref
\begin{bmatrix}
\leading{1} & -\frac{1}{2}\\
0 & 0
\end{bmatrix}\\
\eigenspace{C}{3}&=\nsp{C-(3)I_2}=
\spn{\set{\colvector{\frac{1}{2}\\1}}}
=
\spn{\set{\colvector{1\\2}}}
%
\end{align*}
