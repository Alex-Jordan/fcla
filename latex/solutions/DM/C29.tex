%%%%(c)
%%%%(c)  This file is a portion of the source for the textbook
%%%%(c)
%%%%(c)    A First Course in Linear Algebra
%%%%(c)    Copyright 2004 by Robert A. Beezer
%%%%(c)
%%%%(c)  See the file COPYING.txt for copying conditions
%%%%(c)
%%%%(c)
Expanding along the first column, we have
\begin{align*}
\begin{vmatrix} 
2 & 3 & 0 & 2 & 1\\ 
0 & 1 & 1 & 1 & 2\\
0 & 0 & 1 & 2 & 3\\
0 & 1 & 2 & 1 &0\\
0 & 0 & 0 & 1 & 2
\end{vmatrix}
&= 
2 \begin{vmatrix} 
1 & 1 & 1 & 2\\ 
0 & 1 & 2 & 3\\ 
1 & 2 & 1 & 0\\
0 & 0 & 1 & 2 
\end{vmatrix} 
+ 0 + 0 + 0 + 0\\
\intertext{Now, expanding along the first column again, we have}
&= 2\left( 
\begin{vmatrix}  
1 & 2 & 3\\ 
2 & 1 & 0\\ 
0 & 1 & 2
\end{vmatrix} - 0  + 
\begin{vmatrix} 
1 & 1 & 2\\
1 & 2 & 3\\
0 & 1 & 2 
\end{vmatrix} - 0 
\right)\\
%&= 2\left( [1 \cdot 1 \cdot 2 + 2 \cdot 0 \cdot 0 + 3 \cdot 2 \cdot 1 - 0 \cdot 1 \cdot 3 - 1 \cdot 0 \cdot 1 - 2 \cdot 2 \cdot 2] + \\
%& \hspace*{0.7in}[1 \cdot 2 \cdot 2 + 1 \cdot 3 \cdot 0 +2 \cdot 1 \cdot 1 -0 \cdot 2 \cdot 2 - 1 \cdot 3 \cdot 1 - 2 \cdot 1 \cdot 1]\right)\\
&= 2([2 +0 + 6 - 0 - 0 - 8] + [4 + 0 + 2 - 0 - 3 - 2])\\
&= 2
\end{align*}