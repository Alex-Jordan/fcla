%%%%(c)
%%%%(c)  This file is a portion of the source for the textbook
%%%%(c)
%%%%(c)    A First Course in Linear Algebra
%%%%(c)    Copyright 2004 by Robert A. Beezer
%%%%(c)
%%%%(c)  See the file COPYING.txt for copying conditions
%%%%(c)
%%%%(c)
We can reformulate the linear system as a vector equality with a matrix-vector product via \acronymref{theorem}{SLEMM}.  The system is then represented by $A\vect{x}=\vect{b}$ where
%
\begin{align*}
A&=
\begin{bmatrix}
1 & -1 & 2 \\
1 & 0 & -2\\
2 & -1 & -1
\end{bmatrix}
&
\vect{b}
&=\colvector{5\\-8\\-6}
\end{align*}
%
According to \acronymref{theorem}{SNCM}, if $A$ is nonsingular then the (unique) solution will be given by $\inverse{A}\vect{b}$.  We attempt the computation of $\inverse{A}$ through \acronymref{theorem}{CINM}, or with our favorite computational device and obtain,
%
\begin{align*}
\inverse{A}=
\begin{bmatrix}
 2 & 3 & -2 \\
 3 & 5 & -4 \\
 1 & 1 & -1
\end{bmatrix}
%
\end{align*}
%
So by \acronymref{theorem}{NI}, we know $A$ is nonsingular, and so the unique solution is
%
\begin{align*}
\inverse{A}\vect{b}
=
\begin{bmatrix}
 2 & 3 & -2 \\
 3 & 5 & -4 \\
 1 & 1 & -1
\end{bmatrix}
\colvector{5\\-8\\-6}
=
\colvector{-2\\-1\\3}
\end{align*}
