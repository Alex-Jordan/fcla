%%%%(c)
%%%%(c)  This file is a portion of the source for the textbook
%%%%(c)
%%%%(c)    A First Course in Linear Algebra
%%%%(c)    Copyright 2004 by Robert A. Beezer
%%%%(c)
%%%%(c)  See the file COPYING.txt for copying conditions
%%%%(c)
%%%%(c)
In each case, begin with a vector equation where one side contains a linear combination of the two vectors from the span construction that gives the column space of $A$ with unknowns for scalars, and then use \acronymref{theorem}{SLSLC} to set up a system of equations.  For $\vect{c}$, the corresponding system has no solution, as we would expect.\par
%
For $\vect{b}$ there is a solution, as we would expect.  What is interesting is that the solution is unique.  This is a consequence of the linear independence of the set of two vectors in the span construction.  If we wrote $\vect{b}$ as a linear combination of all four columns of $A$, then there would be infinitely many ways to do this.