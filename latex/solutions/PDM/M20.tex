%%%%(c)
%%%%(c)  This file is a portion of the source for the textbook
%%%%(c)
%%%%(c)    A First Course in Linear Algebra
%%%%(c)    Copyright 2004 by Robert A. Beezer
%%%%(c)
%%%%(c)  See the file COPYING.txt for copying conditions
%%%%(c)
%%%%(c)
The result of these computations should be a matrix with the value of $\detname{A}$ in the diagonal entries and zeros elsewhere.  The suggestion of using a nonsingular matrix was partially so that it was obvious that the value of the determinant appears on the diagonal.\par
%
This result (which is true in general) provides a method for computing the inverse of a nonsingular matrix.  Since $A\transpose{C}=\detname{A}I_n$, we can multiply by the reciprocal of the determinant (which is nonzero!) and the inverse of $A$ (it exists!) to arrive at an expression for the matrix inverse:
%
\begin{equation*}
\inverse{A}=\frac{1}{\detname{A}}\transpose{C}
\end{equation*}
%
