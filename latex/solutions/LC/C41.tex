%%%%(c)
%%%%(c)  This file is a portion of the source for the textbook
%%%%(c)
%%%%(c)    A First Course in Linear Algebra
%%%%(c)    Copyright 2004 by Robert A. Beezer
%%%%(c)
%%%%(c)  See the file COPYING.txt for copying conditions
%%%%(c)
%%%%(c)
Row-reduce the augmented matrix representing this system, to find
%
\begin{equation*}
\begin{bmatrix}
 \leading{1} & 0 & 3 & -2 & 0 & -1 & 0 & 0 & 3 & 6 \\
 0 & \leading{1} & 2 & -4 & 0 & 3 & 0 & 0 & 2 & -1 \\
 0 & 0 & 0 & 0 & \leading{1} & -2 & 0 & 0 & -1 & 3 \\
 0 & 0 & 0 & 0 & 0 & 0 & \leading{1} & 0 & 4 & 0 \\
 0 & 0 & 0 & 0 & 0 & 0 & 0 & \leading{1} & 2 & -2 \\
 0 & 0 & 0 & 0 & 0 & 0 & 0 & 0 & 0 & 0
\end{bmatrix}
\end{equation*}
%
The system is consistent (no leading one in column 10, \acronymref{theorem}{RCLS}).   $F=\set{3,\,4,\,6,\,9,\,10}$, so the free variables are $x_3,\,x_4,\,x_6$ and $x_9$.  Now apply \acronymref{theorem}{VFSLS} directly, or follow the three-step process of \acronymref{example}{VFS}, \acronymref{example}{VFSAD}, \acronymref{example}{VFSAI}, or \acronymref{example}{VFSAL} to obtain the solution set
%
\begin{equation*}
S=\setparts{
\colvector{ 6\\ -1\\ 0\\ 0\\ 3\\ 0\\ 0\\ -2\\0}+
x_3\colvector{ -3\\ -2\\ 1\\ 0\\ 0\\ 0\\ 0\\ 0\\0}+
x_4\colvector{ 2\\ 4\\ 0\\ 1\\ 0\\ 0\\ 0\\ 0\\0}+
x_6\colvector{ 1\\ -3\\ 0\\ 0\\ 2\\ 1\\ 0\\ 0\\0}+
x_9\colvector{ -3\\ -2\\ 0\\ 0\\ 1\\ 0\\ -4\\ -2\\1}
}{
x_3,\,x_4,\,x_6,\,x_9\in\complex{\null}
}
\end{equation*}
