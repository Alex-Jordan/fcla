%%%%(c)
%%%%(c)  This file is a portion of the source for the textbook
%%%%(c)
%%%%(c)    A First Course in Linear Algebra
%%%%(c)    Copyright 2004 by Robert A. Beezer
%%%%(c)
%%%%(c)  See the file COPYING.txt for copying conditions
%%%%(c)
%%%%(c)
$A$ being similar to $B$ means that there exists an $S$ such that $A=\inverse{S}BS$. So, $B=SA\inverse{S}$ and because $S$, $A$, and $\inverse{S}$ are nonsingular, by \acronymref{theorem}{NPNT}, $B$ is nonsingular.
%
\begin{align*}
\inverse{A}
&= \inverse{\left(\inverse{S}BS\right)}
&&\text{\acronymref{definition}{SIM}}\\
%
&= \inverse{S}\inverse{B}\inverse{\left(\inverse{S}\right)}
&&\text{\acronymref{theorem}{SS}}
%
&= \inverse{S}\inverse{B}S
&&\text{\acronymref{theorem}{MIMI}}
%
\end{align*}
%
Then by \acronymref{definition}{SIM}, $\inverse{A}$ is similar to $\inverse{B}$.