%%%%(c)
%%%%(c)  This file is a portion of the source for the textbook
%%%%(c)
%%%%(c)    A First Course in Linear Algebra
%%%%(c)    Copyright 2004 by Robert A. Beezer
%%%%(c)
%%%%(c)  See the file COPYING.txt for copying conditions
%%%%(c)
%%%%(c)
A calculator will report $\lambda=0$ as an eigenvalue of algebraic multiplicity of 2, and $\lambda=-1$ as an eigenvalue of algebraic multiplicity 2 as well.  Since eigenvalues are roots of the characteristic polynomial (\acronymref{theorem}{EMRCP}) we have the factored version
%
\begin{equation*}
\charpoly{A}{x}=(x-0)^2(x-(-1))^2=x^2(x^2+2x+1)=x^4+2x^3+x^2
\end{equation*}
%
The eigenspaces are then
%
\begin{align*}
%
\lambda&=0\\
A-(0)I_4&=
\begin{bmatrix}
 19 & 25 & 30 & 5 \\
 -23 & -30 & -35 & -5 \\
 7 & 9 & 10 & 1 \\
 -3 & -4 & -5 & -1
\end{bmatrix}
\rref
\begin{bmatrix}
 \leading{1} & 0 & -5 & -5 \\
 0 & \leading{1} & 5 & 4 \\
 0 & 0 & 0 & 0 \\
 0 & 0 & 0 & 0
\end{bmatrix}\\
\eigenspace{A}{0}&=\nsp{C-(0)I_4}=
\spn{\set{\colvector{5\\-5\\1\\0},\,\colvector{5\\-4\\0\\1}}}
%
\end{align*}
%
%
\begin{align*}
%
\lambda&=-1\\
A-(-1)I_4&=
\begin{bmatrix}
 20 & 25 & 30 & 5 \\
 -23 & -29 & -35 & -5 \\
 7 & 9 & 11 & 1 \\
 -3 & -4 & -5 & 0
\end{bmatrix}
\rref
\begin{bmatrix}
 \leading{1} & 0 & -1 & 4 \\
 0 & \leading{1} & 2 & -3 \\
 0 & 0 & 0 & 0 \\
 0 & 0 & 0 & 0
\end{bmatrix}\\
\eigenspace{A}{-1}&=\nsp{C-(-1)I_4}=
\spn{\set{\colvector{1\\-2\\1\\0},\,\colvector{-4\\3\\0\\1}}}
%
\end{align*}
%
Each eigenspace above is described by a spanning set obtained through an application of \acronymref{theorem}{BNS} and so is a basis for the eigenspace.  In each case the dimension, and therefore the geometric multiplicity, is 2.\par
%
For each of the two eigenvalues, the algebraic and geometric multiplicities are equal.  \acronymref{theorem}{DMFE} says that in this situation the matrix is diagonalizable.  We know from \acronymref{theorem}{DC} that when we diagonalize $A$ the diagonal matrix will have the eigenvalues of $A$ on the diagonal (in some order).  So we can claim that
%
\begin{equation*}
D=
\begin{bmatrix}
 0 & 0 & 0 & 0 \\
 0 & 0 & 0 & 0 \\
 0 & 0 & -1 & 0 \\
 0 & 0 & 0 & -1
\end{bmatrix}
\end{equation*}