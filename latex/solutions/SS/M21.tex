%%%%(c)
%%%%(c)  This file is a portion of the source for the textbook
%%%%(c)
%%%%(c)    A First Course in Linear Algebra
%%%%(c)    Copyright 2004 by Robert A. Beezer
%%%%(c)
%%%%(c)  See the file COPYING.txt for copying conditions
%%%%(c)
%%%%(c)
If the columns of the coefficient matrix from \acronymref{archetype}{C} are named $\vect{u}_1,\,\vect{u}_2,\,\vect{u}_3,\,\vect{u}_4$ then we can discover the equation 
%
\begin{equation*}
(-2)\vect{u}_1+(-3)\vect{u}_2+\vect{u}_3+\vect{u}_4=\zerovector
\end{equation*}
%
by building a homogeneous system of equations and viewing a solution to the system as scalars in a linear combination via \acronymref{theorem}{SLSLC}.  This particular vector equation can be rearranged to read
%
\begin{equation*}
\vect{u}_4=(2)\vect{u}_1+(3)\vect{u}_2+(-1)\vect{u}_3
\end{equation*}
%
This can be interpreted to mean that $\vect{u}_4$ is unnecessary in 
$\spn{\set{\vect{u}_1,\,\vect{u}_2,\,\vect{u}_3,\,\vect{u}_4}}$, so that 
%
\begin{equation*}
\spn{\set{\vect{u}_1,\,\vect{u}_2,\,\vect{u}_3,\,\vect{u}_4}}
=
\spn{\set{\vect{u}_1,\,\vect{u}_2,\,\vect{u}_3}}
\end{equation*}
%
If we try to repeat this process and find a linear combination of $\vect{u}_1,\,\vect{u}_2,\,\vect{u}_3$ that equals the zero vector, we will fail.  The required homogeneous system of equations (via \acronymref{theorem}{SLSLC}) has only a trivial solution, which will not provide the kind of equation we need to remove one of the three remaining vectors.