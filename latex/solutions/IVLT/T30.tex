%%%%(c)
%%%%(c)  This file is a portion of the source for the textbook
%%%%(c)
%%%%(c)    A First Course in Linear Algebra
%%%%(c)    Copyright 2004 by Robert A. Beezer
%%%%(c)
%%%%(c)  See the file COPYING.txt for copying conditions
%%%%(c)
%%%%(c)
Since $U$ and $V$ are isomorphic, there is at least one isomorphism between them (\acronymref{definition}{IVS}), say $\ltdefn{T}{U}{V}$.  As such, $T$ is an invertible linear transformation.\par
%
For $\alpha\in\complexes$ define the linear transformation $\ltdefn{S}{V}{V}$ by $\lt{S}{\vect{v}}=\alpha\vect{v}$.  Convince yourself that when $\alpha\neq 0$, $S$ is an invertible linear transformation (\acronymref{definition}{IVLT}).  Then the composition, $\ltdefn{\compose{S}{T}}{U}{V}$, is an invertible linear transformation by \acronymref{theorem}{CIVLT}.  Once convinced that each non-zero value of $\alpha$ gives rise to a different functions for $\compose{S}{T}$, then we have constructed infinitely many isomorphisms from $U$ to $V$.