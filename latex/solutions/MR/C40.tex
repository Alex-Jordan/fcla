%%%%(c)
%%%%(c)  This file is a portion of the source for the textbook
%%%%(c)
%%%%(c)    A First Course in Linear Algebra
%%%%(c)    Copyright 2004 by Robert A. Beezer
%%%%(c)
%%%%(c)  See the file COPYING.txt for copying conditions
%%%%(c)
%%%%(c)
The analysis of $R$ will be easiest if we analyze a matrix representation of $R$.  Since we can use any matrix representation, we might as well use natural bases that allow us to construct the matrix representation quickly and easily,
%
\begin{align*}
B&=\set{
\begin{bmatrix}1&0\\0&0\end{bmatrix},\,
\begin{bmatrix}0&1\\1&0\end{bmatrix},\,
\begin{bmatrix}0&0\\0&1\end{bmatrix}
}
&
C&=\set{1,\,x,\,x^2}
\end{align*}
%
then we can practically build the matrix representation on sight,
%
\begin{equation*}
\matrixrep{R}{B}{C}=
\begin{bmatrix}
1 & -1 & 0\\
2 & -3 & -2\\
1 & -1 & 1
\end{bmatrix}
\end{equation*}
%
This matrix representation is invertible (it has a nonzero determinant of $-1$, \acronymref{theorem}{SMZD}, \acronymref{theorem}{NI}) so \acronymref{theorem}{IMR} tells us that the linear transformation $R$ is also invertible.  To find a formula for $\ltinverse{R}$  we compute,
%
\begin{align*}
\lt{\ltinverse{R}}{a+bx+cx^2}
&=\vectrepinv{B}{\matrixrep{\ltinverse{R}}{C}{B}\vectrep{C}{a+bx+cx^2}}&&\text{\acronymref{theorem}{FTMR}}\\
&=\vectrepinv{B}{\inverse{\left(\matrixrep{R}{B}{C}\right)}\vectrep{C}{a+bx+cx^2}}&&\text{\acronymref{theorem}{IMR}}\\
&=\vectrepinv{B}{\inverse{\left(\matrixrep{R}{B}{C}\right)}\colvector{a\\b\\c}}&&\text{\acronymref{definition}{VR}}\\
&=\vectrepinv{B}{
\begin{bmatrix}5&-1&-2\\4&-1&-2\\-1&0&1\end{bmatrix}
\colvector{a\\b\\c}}&&\text{\acronymref{definition}{MI}}\\
&=\vectrepinv{B}{\colvector{5a-b-2c\\4a-b-2c\\-a+c}}&&\text{\acronymref{definition}{MVP}}\\
&=\begin{bmatrix}5a-b-2c&4a-b-2c\\4a-b-2c&-a+c\end{bmatrix}&&\text{\acronymref{definition}{VR}}
\end{align*}
