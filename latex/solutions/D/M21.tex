%%%%(c)
%%%%(c)  This file is a portion of the source for the textbook
%%%%(c)
%%%%(c)    A First Course in Linear Algebra
%%%%(c)    Copyright 2004 by Robert A. Beezer
%%%%(c)
%%%%(c)  See the file COPYING.txt for copying conditions
%%%%(c)
%%%%(c)
A typical matrix from $UT_2$ looks like
%
\begin{equation*}
\begin{bmatrix}
a & b \\ 0 & c
\end{bmatrix}
\end{equation*}
%
where $a,\,b,\,c\in\complex{}$ are arbitrary scalars.  Observing this we can then write
%
\begin{equation*}
%
\begin{bmatrix}
a & b \\ 0 & c
\end{bmatrix}
=
a
\begin{bmatrix}
1 & 0 \\ 0 & 0
\end{bmatrix}
+
b
\begin{bmatrix}
0 & 1 \\ 0 & 0
\end{bmatrix}
+
c
\begin{bmatrix}
0 & 0 \\ 0 & 1
\end{bmatrix}
%
\end{equation*}
%
which says that 
%
\begin{equation*}
R=\set{
\begin{bmatrix}
1 & 0 \\ 0 & 0
\end{bmatrix}
,\,
\begin{bmatrix}
0 & 1 \\ 0 & 0
\end{bmatrix}
,\,\begin{bmatrix}
0 & 0 \\ 0 & 1
\end{bmatrix}
}
\end{equation*}
%
is a spanning set for $UT_2$ (\acronymref{definition}{TSVS}).  Is $R$ is linearly independent?  If so, it is a basis for $UT_2$.  So consider a relation of linear dependence on $R$,
\begin{equation*}
%
\alpha_1
\begin{bmatrix}
1 & 0 \\ 0 & 0
\end{bmatrix}
+
\alpha_2
\begin{bmatrix}
0 & 1 \\ 0 & 0
\end{bmatrix}
+
\alpha_3
\begin{bmatrix}
0 & 0 \\ 0 & 1
\end{bmatrix}
=
\zeromatrix
=
\begin{bmatrix}
0 & 0 \\ 0 & 0
\end{bmatrix}
%
\end{equation*}
%
From this equation, one rapidly arrives at the conclusion that $\alpha_1=\alpha_2=\alpha_3=0$.  So $R$ is a linearly independent set (\acronymref{definition}{LI}), and hence is a basis (\acronymref{definition}{B}) for $UT_2$.  Now, we simply count up the size of the set $R$ to see that the dimension of $UT_2$ is $\dimension{UT_2}=3$.