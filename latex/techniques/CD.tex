%%%%(c)
%%%%(c)  This file is a portion of the source for the textbook
%%%%(c)
%%%%(c)    A First Course in Linear Algebra
%%%%(c)    Copyright 2004 by Robert A. Beezer
%%%%(c)
%%%%(c)  See the file COPYING.txt for copying conditions
%%%%(c)
%%%%(c)
\begin{para}Another proof technique is known as ``proof by contradiction'' and it can be a powerful (and satisfying) approach.  Simply put, suppose you wish to prove the implication, ``If $A$, then $B$.''   As usual, we assume that $A$ is true, but we also make the additional assumption that $B$ is false.  If our original implication is true, then these twin assumptions should lead us to a logical inconsistency.  In practice we assume the negation of $B$ to be true (see \acronymref{technique}{N}).  So we argue from the assumptions $A$ and $\text{not}(B)$ looking for some obviously false conclusion such as $1=6$, or a set is simultaneously empty and nonempty, or a matrix is both nonsingular and singular.\end{para}
%
\begin{para}You should be careful about formulating proofs that look like proofs by contradiction, but really aren't.  This happens when you assume $A$ and $\text{not}(B)$ and proceed to give a ``normal'' and direct proof that $B$ is true by only using the assumption that $A$ is true.  Your last step is to then claim that $B$ is true and you then appeal to the assumption that $\text{not}(B)$ is true, thus getting the desired contradiction.  Instead, you could have avoided the overhead of a proof by contradiction and just run with the direct proof.  This stylistic flaw is known, quite graphically, as ``setting up the strawman to knock him down.''\end{para}
%
\begin{para}Here is a simple example of a proof by contradiction.  There are direct proofs that are just about as easy, but this will demonstrate the point, while narrowly avoiding knocking down the straw man.\end{para}
%
\begin{para}Theorem:  If $a$ and $b$ are odd integers, then their product, $ab$, is odd.\end{para}
%
\begin{para}Proof:  To begin a proof by contradiction, assume the hypothesis, that $a$ and $b$ are odd.  Also assume the negation of the conclusion, in this case, that $ab$ is even.  Then there are integers, $j$, $k$, $\ell$ so that $a=2j+1$, $b=2k+1$, $ab=2\ell$.  Then
%
\begin{align*}
0
&=ab-ab\\
&=(2j+1)(2k+1)-(2\ell)\\
&=4jk+2j+2k-2\ell+1\\
&=2\left(2jk+j+k-\ell\right)+1\\
\end{align*}
\end{para}
%
%
\begin{para}Again, we do not offer this example as the {\em best} proof of this fact about even and odd numbers, but rather it is a simple illustration of a proof by contradiction.  You can find examples of proofs by contradiction in
% RREF is unique
% Nonsingular matrices and unique solutions
% Nonsingular product, nonsingular terms
% 2x2 matrix inverse
% Gram-Schmidt
% Extending linearly independent sets
% Equal dimensions -> equal subspaces
% Every matrix has an eigenvalue
% Eigenvectors with distinct eigenvalues
% Diagonalizable matrices, large eigenspace
\acronymref{theorem}{RREFU},
\acronymref{theorem}{NMUS},
\acronymref{theorem}{NPNT},
\acronymref{theorem}{TTMI},
\acronymref{theorem}{GSP},
\acronymref{theorem}{ELIS},
\acronymref{theorem}{EDYES},
\acronymref{theorem}{EMHE},
\acronymref{theorem}{EDELI},
and
\acronymref{theorem}{DMFE},
%
in addition to several examples and solutions to exercises.\end{para}

