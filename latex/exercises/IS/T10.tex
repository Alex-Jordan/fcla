%%%%(c)
%%%%(c)  This file is a portion of the source for the textbook
%%%%(c)
%%%%(c)    A First Course in Linear Algebra
%%%%(c)    Copyright 2004 by Robert A. Beezer
%%%%(c)
%%%%(c)  See the file COPYING.txt for copying conditions
%%%%(c)
%%%%(c)
Suppose that $\ltdefn{T}{V}{V}$ is linear transformation, and $p(x)$ is a polynomial.  Then define the new linear transformation $\ltdefn{p(T)}{V}{V}$ by interpreting the coefficients of the terms of the polynomial as scalar mutliples of linear transformations (\acronymref{definition}{LTSM}), addition of terms as the sum of linear transformations (\acronymref{definition}{LTA}), and powers as repeated composition of linear transformations (\acronymref{definition}{LTC}).  Prove that $\compose{T}{p(T)}=\compose{p(T)}{T}$.\par
%
Use this observation to give a shorter argument for the proof of the invariance of the generalized eigenspace in \acronymref{theorem}{GESIS}.