%%%%(c)
%%%%(c)  This file is a portion of the source for the textbook
%%%%(c)
%%%%(c)    A First Course in Linear Algebra
%%%%(c)    Copyright 2004 by Robert A. Beezer
%%%%(c)
%%%%(c)  See the file COPYING.txt for copying conditions
%%%%(c)
%%%%(c)
Matrix multiplication interacts nicely with many operations.  But not always with transforming a matrix to reduced row-echelon form.  Suppose that $A$ is an $m\times n$ matrix and $B$ is an $n\times p$ matrix.  Let $P$ be a matrix that is row-equivalent to $A$ and in reduced row-echelon form, $Q$ be a matrix that is row-equivalent to $B$ and in reduced row-echelon form, and let $R$ be a matrix that is row-equivalent to $AB$ and in reduced row-echelon form.  Is $PQ=R$?  (In other words, with nonstandard notation, is $\text{rref}(A)\text{rref}(B)=\text{rref}(AB)$?)\par
%
Construct a counterexample to show that, in general, this statement is false.  Then find a large class of matrices where if $A$ and $B$ are in the class, then the statement is true.