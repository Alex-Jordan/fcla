%%%%(c)
%%%%(c)  This file is a portion of the source for the textbook
%%%%(c)
%%%%(c)    A First Course in Linear Algebra
%%%%(c)    Copyright 2004 by Robert A. Beezer
%%%%(c)
%%%%(c)  See the file COPYING.txt for copying conditions
%%%%(c)
%%%%(c)
We have a variety of theorems about how to create column spaces and row spaces and they frequently involve row-reducing a matrix.  Here is a procedure that some try to use to get a column space.  Begin with an $m\times n$ matrix $A$ and row-reduce to a matrix $B$ with columns $\vectorlist{B}{n}$.  Then form the column space of $A$ as
%
\begin{equation*}
\csp{A}=
\spn{\set{\vectorlist{B}{n}}}
=\csp{B}
\end{equation*}
%
This is {\em not} not a legitimate procedure, and therefore is {\em not} a theorem.  Construct an example to show that the procedure will not in general create the column space of $A$.