%%%%(c)
%%%%(c)  This file is a portion of the source for the textbook
%%%%(c)
%%%%(c)    A First Course in Linear Algebra
%%%%(c)    Copyright 2004 by Robert A. Beezer
%%%%(c)
%%%%(c)  See the file COPYING.txt for copying conditions
%%%%(c)
%%%%(c)
%%%%%%%%%%%
%%
%%  Annotated Acronyms SLE
%%  Systems of Linear Equations
%%
%%%%%%%%%%%
%
At the conclusion of each chapter you will find a section like this, reviewing selected definitions and theorems.  There are many reasons for why a definition or theorem might be placed here.  It might represent a key concept, it might be used frequently for computations, provide the critical step in many proofs, or it may deserve special comment.\par
%
These lists are not meant to be exhaustive, but should still be useful as part of reviewing each chapter.  We will mention a few of these that you might eventually recognize on sight as being worth memorization.  By that we mean that you can associate the acronym with a rough statement of the theorem --- not that the exact details of the theorem need to be memorized.  And it is certainly not our intent that everything on these lists is important enough to memorize.\par
\bigskip
%
\annoacro{theorem}{RCLS}{%
We will repeatedly appeal to this theorem to determine if a system of linear equations, does, or doesn't, have a solution.  This one we will see often enough that it is worth memorizing.}
%
\annoacro{theorem}{HMVEI}{%
This theorem is the theoretical basis of several of our most important theorems.  So keep an eye out for it, and its descendants, as you study other proofs.  For example, \acronymref{theorem}{HMVEI} is critical to the proof of \acronymref{theorem}{SSLD}, \acronymref{theorem}{SSLD} is critical to the proof of \acronymref{theorem}{G},  \acronymref{theorem}{G} is critical to the proofs of the pair of similar theorems, \acronymref{theorem}{ILTD} and \acronymref{theorem}{SLTD}, while finally \acronymref{theorem}{ILTD} and \acronymref{theorem}{SLTD} are critical to the proof of an important result, \acronymref{theorem}{IVSED}.  This chain of implications might not make much sense on a first reading, but come back later to see how some very important theorems build on the seemingly simple result that is \acronymref{theorem}{HMVEI}.  Using the ``find'' feature in whatever software you use to read the electronic version of the text can be a fun way to explore these relationships.
}
%
\annoacro{theorem}{NMRRI}{%
This theorem gives us one of simplest ways, computationally, to recognize if a matrix is nonsingular, or singular.  We will see this one often, in computational exercises especially.}
%
\annoacro{theorem}{NMUS}{%
Nonsingular matrices will be an important topic going forward (witness the NMEx series of theorems).  This is our first result along these lines, a useful theorem for other proofs, and also illustrates a more general concept from \acronymref{chapter}{LT}.  
}
%
% End SLE.tex annotated acronyms
