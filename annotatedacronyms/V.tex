%%%%(c)
%%%%(c)  This file is a portion of the source for the textbook
%%%%(c)
%%%%(c)    A First Course in Linear Algebra
%%%%(c)    Copyright 2004 by Robert A. Beezer
%%%%(c)
%%%%(c)  See the file COPYING.txt for copying conditions
%%%%(c)
%%%%(c)
%%%%%%%%%%%
%%
%%  Annotated Acronyms V
%%  Vectors
%%
%%%%%%%%%%%
%
\annoacro{theorem}{VSPCV}{%
These are the fundamental rules for working with the addition, and scalar multiplication, of column vectors.  We will see something very similar in the next chapter (\acronymref{theorem}{VSPM}) and then this will be generalized into what is arguably our most important definition, \acronymref{definition}{VS}.
}
%
\annoacro{theorem}{SLSLC}{%
Vector addition and scalar multiplication are the two fundamental operations on vectors, and linear combinations roll them both into one.  \acronymref{theorem}{SLSLC} connects linear combinations with systems of equations.  This one we will see often enough that it is worth memorizing.
}
%
\annoacro{theorem}{PSPHS}{%
This theorem is interesting in its own right, and sometimes the vaugeness surrounding the choice of $\vect{z}$ can seem mysterious.  But we list it here because we will see an important theorem in \acronymref{section}{ILT} which will generalize this result (\acronymref{theorem}{KPI}).
}
%
\annoacro{theorem}{LIVRN}{%
If you have a set of column vectors, this is the fastest computational approach to determine if the set is linearly independent.  Make the vectors the columns of a matrix, row-reduce, compare $r$ and $n$.  That's it --- and you always get an answer.  Put this one in your toolkit.
}
%
\annoacro{theorem}{BNS}{%
We will have several theorems (all listed in these ``Annotated Acronyms'' sections) whose conclusions will provide a linearly independent set of vectors whose span equals some set of interest (the null space here).  While the notation in this theorem might appear gruesome, in practice it can become very routine to apply.  So practice this one --- we'll be using it all through the book.
}
%
\annoacro{theorem}{BS}{%
As promised, another theorem that provides a linearly independent set of vectors whose span equals some set of interest (a span now).  You can use this one to clean up {\em any} span.
}
%
% End V.tex annotated acronyms
