%%%%(c)
%%%%(c)  This file is a portion of the source for the textbook
%%%%(c)
%%%%(c)    A First Course in Linear Algebra
%%%%(c)    Copyright 2004 by Robert A. Beezer
%%%%(c)
%%%%(c)  See the file COPYING.txt for copying conditions
%%%%(c)
%%%%(c)
%%%%%%%%%%%
%%
%%  Annotated Acronyms VS
%%  Vector Spaces
%%
%%%%%%%%%%%
%
\annoacro{definition}{VS}{%
The most fundamental object in linear algebra is a vector space.  Or else the most fundamental object is a vector, and a vector space is important because it is a collection of vectors.  Either way, \acronymref{definition}{VS} is critical.  All of our remaining theorems that assume we are working with a vector space can trace their lineage back to this definition.
}
%
\annoacro{theorem}{TSS}{%
Check all ten properties of a vector space (\acronymref{definition}{VS}) can get tedious.  But if you have a subset of a {\em known} vector space, then \acronymref{theorem}{TSS} considerably shortens the verification.  Also, proofs of closure (the last two conditions in \acronymref{theorem}{TSS}) are a good way to practice a common style of proof.
}
%
\annoacro{theorem}{VRRB}{%
The proof of uniqueness in this theorem is a very typical employment of the hypothesis of linear independence.  But that's not why we mention it here.  This theorem is critical to our first section about representations, \acronymref{section}{VR}, via \acronymref{definition}{VR}.
}
%
\annoacro{theorem}{CNMB}{%
Having just defined a basis (\acronymref{definition}{B}) we discover that the columns of a nonsingular matrix form a basis of $\complex{m}$.  Much of what we know about nonsingular matrices is either contained in this statement, or much more evident because of it.
}
%
\annoacro{theorem}{SSLD}{%
This theorem is a key juncture in our development of linear algebra.  You have probably already realized how useful \acronymref{theorem}{G} is.  All four parts of \acronymref{theorem}{G} have proofs that finish with an application of \acronymref{theorem}{SSLD}.
}
%
\annoacro{theorem}{RPNC}{%
This simple relationship between the rank, nullity and number of columns of a matrix might be surprising.  But in simplicity comes power, as this theorem can be very useful.  It will be generalized in the very last theorem of \acronymref{chapter}{LT}, \acronymref{theorem}{RPNDD}.
}
%
\annoacro{theorem}{G}{%
A whimsical title, but the intent is to make sure you don't miss this one.  Much of the interaction between bases, dimension, linear independence and spanning is captured in this theorem.
}
%
\annoacro{theorem}{RMRT}{%
This one is a real surprise.  Why should a matrix, and its transpose, both row-reduce to the same number of non-zero rows?
}
%
% End VS.tex annotated acronyms
