%%%%(c)
%%%%(c)  This file is a portion of the source for the textbook
%%%%(c)
%%%%(c)    A First Course in Linear Algebra
%%%%(c)    Copyright 2004 by Robert A. Beezer
%%%%(c)
%%%%(c)  See the file COPYING.txt for copying conditions
%%%%(c)
%%%%(c)
{\sl Mathematica} will solve a linear system of equations using the \computerfont{LinearSolve[\,]} command.  The inputs are a matrix with the coefficients of the variables (but not the column of constants), and a list containing the constant terms of each equation.  This will look a bit odd, since the lists in the matrix are rows, but the column of constants is also input as a list and so looks like a row rather than a column.  The result will be a single solution (even if there are infinitely many), reported as a list, or the statement that there is no solution.  When there are infinitely many, the single solution reported is exactly that solution used in the proof of \acronymref{theorem}{RCLS}, where the free variables are all set to zero, and the dependent variables come along with values from the final column of the row-reduced matrix.\par
%
As an example, \acronymref{archetype}{A} is 
%
\archetypepart{A}{definition}
%
To ask {\sl Mathematica} for a solution, enter
%
\begin{equation*}
\computerfont{
LinearSolve[\ \{\{1,\,-1,\,2\},\{2,\,1,\,1\},\{1,\,1,\,0\}\},\ \{1,\,8,\,5\}\ ]
}
\end{equation*}
and you will get back the single solution
%
\begin{equation*}
\{3,\,2,\,0\}
\end{equation*}
%
We will see later how to coax {\sl Mathematica} into giving us infinitely many solutions for this system (\acronymref{computation}{VFSS.MMA}).
