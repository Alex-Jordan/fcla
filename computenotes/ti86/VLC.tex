%%%%(c)
%%%%(c)  This file is a portion of the source for the textbook
%%%%(c)
%%%%(c)    A First Course in Linear Algebra
%%%%(c)    Copyright 2004 by Robert A. Beezer
%%%%(c)
%%%%(c)  See the file COPYING.txt for copying conditions
%%%%(c)
%%%%(c)
\contributedby{\robertbeezer}\\
Vector operations on the TI-86 can be accessed via the \computerfont{VECTR} key, which is \computerfont{Yellow-8}.  The \computerfont{EDIT} tool appears when the \computerfont{F2} key is pressed.  After providing a name and giving a ``dimension'' (the size) then you can enter the individual entries, one at a time.  Vectors can also be entered on the home screen using brackets (\computerfont{[},\,\computerfont{]}).  To create the vector
%
\begin{equation*}
\vect{v}=\colvector{1\\2\\3\\4}
\end{equation*}
%
use brackets and the store key (\computerfont{STO}),
%
\begin{equation*}
\computerfont{[1,\,2,\,3,\,4]\rightarrow v}
\end{equation*}
%
Vector addition and scalar multiplication are then very natural.  If \computerfont{u} and \computerfont{v} are two vectors of equal size, then
%
\begin{equation*}
\computerfont{2*u + (-3)*v}
\end{equation*}
%
will compute the correct vector and display the result as a vector. 