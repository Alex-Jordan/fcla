%%%%(c)
%%%%(c)  This file is a portion of the source for the textbook
%%%%(c)
%%%%(c)    A First Course in Linear Algebra
%%%%(c)    Copyright 2004 by Robert A. Beezer
%%%%(c)
%%%%(c)  See the file COPYING.txt for copying conditions
%%%%(c)
%%%%(c)
\contributedby{\robertbeezer}\\
Vectors in {\sl SAGE} are constructed from lists, and are displayed horizontally.  For example, the vector
%
\begin{equation*}
\vect{v}=\colvector{1\\2\\3\\4}
\end{equation*}
%
would be entered and named via the command
%
\begin{equation*}
\computerfont{v=vector(QQ, [1,\,2,\,3,\,4])}
\end{equation*}
%
See the notes about rings (\acronymref{computation}{R.SAGE}) and matrix entry (\acronymref{computation}{ME.SAGE}) for reminders about specifying the relevant ring.\par
%
Vector addition and scalar multiplication are then very natural.  If \computerfont{u} and \computerfont{v} are two vectors of the same size, then
%
\begin{equation*}
\computerfont{2 * u + (-3) * v}
\end{equation*}
%
will compute the correct vector.  The result can be assigned to a variable (which will then contain a vector), or be printed.  If printed, it will be written horizontally with parentheses for grouping.  If \computerfont{u} and \computerfont{v} have different sizes, then {\sl SAGE} will complain about ``unsupported operand(s).''
