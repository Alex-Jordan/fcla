%%%%(c)
%%%%(c)  This file is a portion of the source for the textbook
%%%%(c)
%%%%(c)    A First Course in Linear Algebra
%%%%(c)    Copyright 2004 by Robert A. Beezer
%%%%(c)
%%%%(c)  See the file COPYING.txt for copying conditions
%%%%(c)
%%%%(c)
%
\exercise{C10}{\chrisblack}        % basic computations
\exercise{C11}{\chrisblack}
\exercise{C12}{\chrisblack}
\exercise{C13}{\chrisblack}
\exercise{C14}{\chrisblack}
%
\begin{exercisegroup}
\begin{para}In \acronymref{chapter}{V} we defined the operations of vector addition and vector  scalar multiplication in \acronymref{definition}{CVA} and \acronymref{definition}{CVSM}.  These two operations formed the underpinnings of the remainder of the chapter.  We have now defined similar operations for matrices in \acronymref{definition}{MA} and \acronymref{definition}{MSM}.  You will have noticed the resulting similarities between \acronymref{theorem}{VSPCV} and \acronymref{theorem}{VSPM}.\end{para}
%
\begin{para}In Exercises M20--M25, you will be asked to extend these similarities to other fundamental definitions and concepts we first saw in \acronymref{chapter}{V}.  This sequence of problems was suggested by \martinjackson.\end{para}
%
\exercise{M20}{\robertbeezer}   % Definitions
\exercise{M21}{\robertbeezer}   % Standard basis is lin ind
\exercise{M22}{\robertbeezer}   % Stray lin ind set in M_23
\exercise{M23}{\robertbeezer}   % Span question
\exercise{M24}{\robertbeezer}   % Symmetric matrices, basis
\exercise{M25}{\robertbeezer}   % Upper triangular, basis
\end{exercisegroup}
%
%                                              % 10 axioms to prove, number consecutively
\exercise{T13}{\robertbeezer}   % CM
\exercise{T14}{\robertbeezer}   % AAM
\exercise{T17}{\robertbeezer}   % SMAM
\exercise{T18}{\robertbeezer}   % DVAM
%
\begin{exercisegroup}
\begin{para}A matrix $A$ is \define{skew-symmetric} if $\transpose{A}=-A$  Exercises T30--T37 employ this definition.\end{para}
\exercise{T30}{\robertbeezer}   % skew => square
\exercise{T31}{\manleyperkel}   % skew => zero diagonal
\exercise{T32}{\manleyperkel}   % skew, symmetric iff zero matrix
\exercise{T33}{\manleyperkel}   % linear combo of skew is skew
\exercise{T34}{\manleyperkel}   % A + A-transpose is symmetric
\exercise{T35}{\manleyperkel}   % A - A-transpose is skew
\exercise{T36}{\manleyperkel}   % Decompose to symmetric + skew 
\exercise{T37}{\manleyperkel}   % Decomposition is unique 
\end{exercisegroup}






