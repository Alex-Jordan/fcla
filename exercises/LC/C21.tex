%%%%(c)
%%%%(c)  This file is a portion of the source for the textbook
%%%%(c)
%%%%(c)    A First Course in Linear Algebra
%%%%(c)    Copyright 2004 by Robert A. Beezer
%%%%(c)
%%%%(c)  See the file COPYING.txt for copying conditions
%%%%(c)
%%%%(c)
Consider each archetype that is a system of equations. For individual solutions listed (both for the original system and the corresponding homogeneous system) express the vector of constants as a linear combination of the columns of the coefficient matrix, as guaranteed by \acronymref{theorem}{SLSLC}.  Verify this equality by computing the linear combination.  For systems with no solutions, recognize that it is then impossible to write the vector of constants as a linear combination of the columns of the coefficient matrix.  Note too, for homogeneous systems, that the solutions give rise to linear combinations that equal the zero vector.\\
\acronymref{archetype}{A}\\ 
\acronymref{archetype}{B}\\ 
\acronymref{archetype}{C}\\ 
\acronymref{archetype}{D}\\ 
\acronymref{archetype}{E}\\ 
\acronymref{archetype}{F}\\ 
\acronymref{archetype}{G}\\ 
\acronymref{archetype}{H}\\ 
\acronymref{archetype}{I}\\
\acronymref{archetype}{J}

