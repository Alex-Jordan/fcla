%%%%(c)
%%%%(c)  This file is a portion of the source for the textbook
%%%%(c)
%%%%(c)    A First Course in Linear Algebra
%%%%(c)    Copyright 2004 by Robert A. Beezer
%%%%(c)
%%%%(c)  See the file COPYING.txt for copying conditions
%%%%(c)
%%%%(c)
Consider the matrix $A$ below.  First, show that $A$ is diagonalizable by computing the geometric multiplicities of the eigenvalues and quoting the relevant theorem.  Second, find a diagonal matrix $D$ and a nonsingular matrix $S$ so that $\similar{A}{S}=D$.  (See \acronymref{exercise}{EE.C20} for some of the necessary computations.)
%
\begin{equation*}
A=
\begin{bmatrix}
18 & -15 & 33 & -15\\
-4 & 8 & -6 & 6\\
-9 & 9 & -16 & 9\\
5 & -6 & 9 & -4
\end{bmatrix}
\end{equation*}
