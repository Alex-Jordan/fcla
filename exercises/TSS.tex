%%%%(c)
%%%%(c)  This file is a portion of the source for the textbook
%%%%(c)
%%%%(c)    A First Course in Linear Algebra
%%%%(c)    Copyright 2004 by Robert A. Beezer
%%%%(c)
%%%%(c)  See the file COPYING.txt for copying conditions
%%%%(c)
%%%%(c)
%
%
\exercise{C10}{\robertbeezer}   % Infinite solutions in Archetype J
%
%
%
For Exercises C21--C28, find the solution set of the given system of linear equations. Identify the values of $n$ and $r$, and compare your answers to the results of the theorems of this section.\\
\exercise{C21}{\chrisblack}     % 3 x 4 system; no solutions
\exercise{C22}{\chrisblack}     % 3 x 4 system; infinite solutions
\exercise{C23}{\chrisblack}     % 3 x 4 system; no solutions
\exercise{C24}{\chrisblack}     % 3 x 4 system; infinite solutions -- 2 free variables
\exercise{C25}{\chrisblack}     % 4 x 3 system; no solutions
\exercise{C26}{\chrisblack}     % 4 x 3 system; one solution
\exercise{C27}{\chrisblack}     % 4 x 3 system; no solutions
\exercise{C28}{\chrisblack}     % 4 x 3 system; infinite solutions
%%
\exercise{M45}{\robertbeezer}   % infinite solutions in Archetype J, no rref
\exercise{M46}{\manleyperkel}   % various entries of row-reduced Archetype J
%
For Exercises M51--M57  say {\bf as much as possible} about each system's solution set.  Be sure to make it clear which theorems you are using to reach your conclusions.\\
\exercise{M51}{\robertbeezer}  % Vars + equations + extra, what can be said??
\exercise{M52}{\robertbeezer}
\exercise{M53}{\robertbeezer}
\exercise{M54}{\robertbeezer}
%%%  Slot an M55 problem in here eventually
\exercise{M56}{\robertbeezer}
\exercise{M57}{\robertbeezer}
%
\exercise{M60}{\robertbeezer}  % Say as much as possible about archetypes
\exercise{M70}{\manleyperkel}  % Generalized, somewhat, TSS.M46
%
\exercise{T10}{\robertbeezer}   % n-r is negative?
\exercise{T20}{\manleyperkel}   % full generalization of TSS.M46, TSS.M70
\exercise{T40}{\robertbeezer}   % consistent, equal columns -> infinite solutions
\exercise{T41}{\robertbeezer}   % matrix column multiple of constants -> consistent
