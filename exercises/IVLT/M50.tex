%%%%(c)
%%%%(c)  This file is a portion of the source for the textbook
%%%%(c)
%%%%(c)    A First Course in Linear Algebra
%%%%(c)    Copyright 2004 by Robert A. Beezer
%%%%(c)
%%%%(c)  See the file COPYING.txt for copying conditions
%%%%(c)
%%%%(c)
Rework \acronymref{example}{CIVLT}, only in place of the basis $B$ for $P_2$, choose instead to use the basis $C=\set{1,\,1+x,\,1+x+x^2}$.  This will complicate writing a generic element of the domain of $\ltinverse{T}$ as a linear combination of the basis elements, and the algebra will be a bit messier, but in the end you should obtain the same formula for $\ltinverse{T}$.  The inverse linear transformation is what it is, and the choice of a particular basis should not influence the outcome.