%%%%(c)
%%%%(c)  This file is a portion of the source for the textbook
%%%%(c)
%%%%(c)    A First Course in Linear Algebra
%%%%(c)    Copyright 2004 by Robert A. Beezer
%%%%(c)
%%%%(c)  See the file COPYING.txt for copying conditions
%%%%(c)
%%%%(c)
\begin{exercisegroup}
\begin{para}In Exercises C30--C33 determine if the matrix is nonsingular or singular.  Give reasons for your answer.\end{para}
\exercise{C30}{\robertbeezer}  % 4x4 matrix nonsingular? Yes
\exercise{C31}{\robertbeezer}  % 4x4 matrix nonsingular? No
\exercise{C32}{\robertbeezer}  % 3x4 matrix nonsingular? Huh?
\exercise{C33}{\robertbeezer}  % 4x4 matrix nonsingular? Yes
\end{exercisegroup}
%
\exercise{C40}{\robertbeezer}  % Remaining square archetypes, nonsingular?
%
\exercise{C50}{\robertbeezer}  % Null space of 4x4 matrix
%
\exercise{M30}{\robertbeezer}  % 4x4 sytem, ns coeff matrix -> unique solution
%
\begin{exercisegroup}
\begin{para}For Exercises M51--M52  say {\bf as much as possible} about each system's solution set.  Be sure to make it clear which theorems you are using to reach your conclusions.\end{para}
\exercise{M51}{\robertbeezer}  % Vars, equations, singula/nonsingular coeff matrix
\exercise{M52}{\robertbeezer}  % 
\end{exercisegroup}
%
\exercise{T10}{\robertbeezer}  % Singular => last row zero in RREF
\exercise{T12}{\robertbeezer}  % Every column a pivot column iff identity matrix
\exercise{T30}{\robertbeezer}  % Row equivalent -> one nonsingular, so is the other
\exercise{T90}{\robertbeezer}  % Nonsingular => uniq solns, no RREF facts used
