%%%%(c)
%%%%(c)  This file is a portion of the source for the textbook
%%%%(c)
%%%%(c)    A First Course in Linear Algebra
%%%%(c)    Copyright 2004 by Robert A. Beezer
%%%%(c)
%%%%(c)  See the file COPYING.txt for copying conditions
%%%%(c)
%%%%(c)
Fill in each blank with an appropriate vector space property to provide justification for the proof of the following proposition:\par
%
{\bf Proposition 3.} 
For any scalar $c$,  $c\,\vect{0} = \vect{0}$.\par
%
{\bf Proof}: Let $c$ be an arbitrary scalar.\par
%
\begin{tabular}{ll}
1.\quad Then $c\,\vect{0} = c\left(\vect{0} + \vect{0}\right)$,&\underline{\hspace*{2.0in}}\\
2.\quad so $c\,\vect{0} = c\, \vect{0} + c\,\vect{0}$.&\underline{\hspace*{2.0in}}\\
3.\quad We then have $c\, \vect{0} + (-c\, \vect{0}) = (c\,\vect{0} + c\, \vect{0}) + (-c\, \vect{0})$,&Additive Property of Equality\\
4.\quad so that $c\, \vect{0} + (-c\,\vect{0}) = c\,\vect{0} + \left(c\, \vect{0} + (-c\,\vect{0})\right)$.&\underline{\hspace*{2.0in}}\\
5.\quad It follows that $\vect{0} = c\,\vect{0} + \vect{0}$,&\underline{\hspace*{2.0in}}\\
6.\quad and finally we have $\vect{0} = c\,\vect{0}$.&\underline{\hspace*{2.0in}}\\
\end{tabular}\par
%
Thus, for any scalar $c$,  $c\,\vect{0} = \vect{0}$.\quad$\square$