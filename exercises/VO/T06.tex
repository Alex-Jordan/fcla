%%%%(c)
%%%%(c)  This file is a portion of the source for the textbook
%%%%(c)
%%%%(c)    A First Course in Linear Algebra
%%%%(c)    Copyright 2004 by Robert A. Beezer
%%%%(c)
%%%%(c)  See the file COPYING.txt for copying conditions
%%%%(c)
%%%%(c)
Fill in each blank with an appropriate vector space property to provide justification for the proof of the following proposition:\par
%
{\bf Proposition 2.} 
For any vector $\vect{u}\in\complex{m}$, $0\vect{u}=\zerovector.$\par
%
{\bf Proof}: Let $\vect{u}\in\complex{m}$.\par
%
\begin{tabular}{ll}
1.\quad Since $0 + 0 = 0$, we have $0\vect{u} = (0 + 0)\vect{u}$.& Substitution\\
2.\quad We then have $0\vect{u} = 0 \vect{u} + 0 \vect{u}$.&\underline{\hspace*{2.0in}}\\
3.\quad It follows that $0\vect{u} + [-(0 \vect{u})] = (0\vect{u} +  0\vect{u}) + [-(0 \vect{u})]$,&Additive Property of Equality\\
4.\quad so $0\vect{u} + [-(0 \vect{u})] = 0\vect{u}+\left(0\vect{u}+[-(0 \vect{u})]\right)$,&\underline{\hspace*{2.0in}}\\
5.\quad so that $\vect{0} = 0\vect{u} + \vect{0}$,&\underline{\hspace*{2.0in}}\\
6.\quad and thus $\vect{0} = 0\vect{u}$.&\underline{\hspace*{2.0in}}\\
\end{tabular}\par
%
Thus, for any vector $\vect{u}\in\complex{m}$,  $0\vect{u}=\zerovector$.\quad$\square$