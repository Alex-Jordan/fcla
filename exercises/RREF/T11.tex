%%%%(c)
%%%%(c)  This file is a portion of the source for the textbook
%%%%(c)
%%%%(c)    A First Course in Linear Algebra
%%%%(c)    Copyright 2004 by Robert A. Beezer
%%%%(c)
%%%%(c)  See the file COPYING.txt for copying conditions
%%%%(c)
%%%%(c)
Suppose that $A$, $B$ and $C$ are $m\times n$ matrices.  Use the definition of row-equivalence (\acronymref{definition}{REM}) to prove the following three facts.
%
\begin{enumerate}
%
\item $A$ is row-equivalent to $A$.
%
\item If $A$ is row-equivalent to $B$, then $B$ is row-equivalent to $A$.
%
\item If $A$ is row-equivalent to $B$, and $B$ is row-equivalent to $C$, then $A$ is row-equivalent to $C$.
%
\end{enumerate}
%
A relationship that satisfies these three properties is known as an \define{equivalence relation}, an important idea in the study of various algebras.  This is a formal way of saying that a relationship behaves like equality, without requiring the relationship to be as strict as equality itself.   We'll see it again in \acronymref{theorem}{SER}.