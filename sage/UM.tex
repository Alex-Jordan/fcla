No surprise about how we check if a matrix is unitary.  Here is \acronymref{example}{UM3},
%
\begin{sageexample}
sage: A = matrix(QQbar, [
...       [(1+I)/sqrt(5), (3+2*I)/sqrt(55), (2+2*I)/sqrt(22)],
...       [(1-I)/sqrt(5), (2+2*I)/sqrt(55),  (-3+I)/sqrt(22)],
...       [    I/sqrt(5), (3-5*I)/sqrt(55),    (-2)/sqrt(22)]
...                     ])
sage: A.is_unitary()
True
sage: A.conjugate_transpose() == A.inverse()
True
\end{sageexample}
%
We can verify \acronymref{theorem}{UMPIP}, where the vectors \verb?u? and \verb?v? are created randomly.  Try evaluating this compute cell with your own choices.
%
\begin{sageexample}
sage: u = random_vector(QQ, 3) + QQbar(I)*random_vector(QQ, 3)
sage: v = random_vector(QQ, 3) + QQbar(I)*random_vector(QQ, 3)
sage: (A*u).hermitian_inner_product(A*v) == u.hermitian_inner_product(v)
True
\end{sageexample}
%
If you want to experiment with permutation matrices, Sage has these too.  We can create a permutation matrix from a list that indicates for each column the row with a one in it.  Notice that the product here of two permutation matrices is again a permutation matrix.
%
\begin{sageexample}
sage: sigma = Permutation([2,3,1])
sage: S = sigma.to_matrix(); S
[0 0 1]
[1 0 0]
[0 1 0]
sage: tau = Permutation([1,3,2])
sage: T = tau.to_matrix(); T
[1 0 0]
[0 0 1]
[0 1 0]
sage: S*T
[0 1 0]
[1 0 0]
[0 0 1]
sage: rho = Permutation([2, 1, 3])
sage: S*T == rho.to_matrix()
True
\end{sageexample}
%
\begin{sageverbatim}
\end{sageverbatim}
%




