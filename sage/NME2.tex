As will be our habit, we can illustrate properties of nonsingular matrices with random square matrices.  Review \acronymref{sage}{NME1} for a refresher on generating these matrices.  Here we will illustrate the fifth condition, half of \acronymref{theorem}{NMLIC}.
%
\begin{sageexample}
sage: n = 6
sage: A = random_matrix(QQ, n, algorithm='unimodular')
sage: A                                  # random
[   2    7   37  -79  268   44]
[  -1   -3  -16   33 -110  -16]
[   1    1    7  -14   44    5]
[   0   -3  -15   34 -118  -23]
[   2    6   33  -68  227   34]
[  -1   -3  -16   35 -120  -21]
\end{sageexample}
%
\begin{sageexample}
sage: V = QQ^n
sage: V.linear_dependence(A.columns()) == []
True
\end{sageexample}
%
You should always get an empty list (\verb?[]?) as the result of the second compute cell, no matter which random (unimodular) matrix you produce prior.  Note that the list of vectors created using \verb?.columns()? is exactly the list we want to feed to \verb?.linear_dependence()?.
%
\begin{sageverbatim}
\end{sageverbatim}
%