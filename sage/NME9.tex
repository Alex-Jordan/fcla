Our final fact about nonsingular matrices expresses the correspondence between invertible matrices and invertible linear transformations.  As a Sage demonstration, we will begin with an invertible linear transformation and examine two matrix representations.  We will create the linear transformation with nonstandard bases and then compute its representation relative to standard bases.
%
\begin{sageexample}
sage: x1, x2, x3, x4 = var('x1, x2, x3, x4')
sage: outputs = [ x1        - 2*x3 - 4*x4,
...                      x2 -   x3 - 5*x4,
...              -x1 - 2*x2 + 2*x3 + 7*x4,
...                  -   x2        +   x4]
sage: T_symbolic(x1, x2, x3, x4) = outputs

sage: b0 = vector(QQ, [ 1, -2, -1,  8])
sage: b1 = vector(QQ, [ 0,  1,  0, -2])
sage: b2 = vector(QQ, [-1, -2,  2, -5])
sage: b3 = vector(QQ, [-1, -3,  2, -2])
sage: U = (QQ^4).subspace_with_basis([b0, b1, b2, b3])

sage: c0 = vector(QQ, [ 3, -1,  4, -8])
sage: c1 = vector(QQ, [ 1,  0,  1, -1])
sage: c2 = vector(QQ, [ 0,  2, -1,  6])
sage: c3 = vector(QQ, [-1,  2, -2,  8])
sage: V = (QQ^4).subspace_with_basis([c0, c1, c2, c3])

sage: T = linear_transformation(U, V, T_symbolic)
sage: T
Vector space morphism represented by the matrix:
[ 131  -56 -321  366]
[ -37   17   89 -102]
[ -61   25  153 -173]
[  -7   -1   24  -25]
Domain: Vector space of degree 4 and dimension 4 over Rational Field
User basis matrix:
[ 1 -2 -1  8]
[ 0  1  0 -2]
[-1 -2  2 -5]
[-1 -3  2 -2]
Codomain: Vector space of degree 4 and dimension 4 over Rational Field
User basis matrix:
[ 3 -1  4 -8]
[ 1  0  1 -1]
[ 0  2 -1  6]
[-1  2 -2  8]
sage: T.is_invertible()
True
sage: T.matrix(side='right').is_invertible()
True
sage: (T^-1).matrix(side='right') == (T.matrix(side='right'))^-1
True
\end{sageexample}
%
We can convert \verb?T? to a new representation using standard bases for \verb?QQ^4? by computing images of the standard basis.
%
\begin{sageexample}
sage: images = [T(u) for u in (QQ^4).basis()]
sage: T_standard = linear_transformation(QQ^4, QQ^4, images)
sage: T_standard
Vector space morphism represented by the matrix:
[ 1  0 -1  0]
[ 0  1 -2 -1]
[-2 -1  2  0]
[-4 -5  7  1]
Domain: Vector space of dimension 4 over Rational Field
Codomain: Vector space of dimension 4 over Rational Field
sage: T_standard.matrix(side='right').is_invertible()
True
\end{sageexample}
%
Understand that \emph{any} matrix representation of \verb?T_symbolic? will have an invertible matrix representation, no matter which bases are used.  If you look at the matrix representation of \verb?T_standard? and the definition of \verb?T_symbolic? the construction of this example will be transparent, especially if you know the random matrix constructor,
%
\begin{sageverbatim}
sage: A = random_matrix(QQ, 4, algorithm='unimodular', upper_bound=9)
sage: A                                                      # random
[-1 -1  2  1]
[ 1  1 -1  0]
[-2 -1  5  6]
[ 1  0 -4 -5]
\end{sageverbatim}
%
\begin{sageverbatim}
\end{sageverbatim}
%









