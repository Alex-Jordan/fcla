For square matrices, Sage has the methods \verb?.is_singular()? and \verb?.is_invertible()?.  By \acronymref{theorem}{NI} we know these two functions to be logical opposites.  One way to express this is that these two methods will always return different values.  Here we demonstrate with a nonsingular matrix and a singular matrix.  The comparison \verb?!=? is ``not equal.''
%
\begin{sageexample}
sage: nonsing = matrix(QQ, [[ 0, -1,  1, -3],
...                         [ 1,  1,  0, -3],
...                         [ 0,  4, -3,  8],
...                         [-2, -4,  1,  5]])
sage: nonsing.is_singular() != nonsing.is_invertible()
True
sage: sing = matrix(QQ, [[ 1, -2, -6,  8],
...                      [-1,  3,  7, -8],
...                      [ 0, -4, -3, -2],
...                      [ 0,  3,  1,  4]])
sage: sing.is_singular() != sing.is_invertible()
True
\end{sageexample}
%
We could test other properties of the matrix inverse, such as \acronymref{theorem}{SS}.
%
\begin{sageexample}
sage: A = matrix(QQ, [[ 3, -5, -2,  8],
...                   [-1,  2,  0, -1],
...                   [-2,  4,  1, -4],
...                   [ 4, -5,  0,  8]])
sage: B = matrix(QQ, [[ 1,  2,  4, -1],
...                   [-2, -3, -8, -2],
...                   [-2, -4, -7,  5],
...                   [ 2,  5,  7, -8]])
sage: A.is_invertible() and B.is_invertible()
True
sage: (A*B).inverse() == B.inverse()*A.inverse()
True
\end{sageexample}
%
\begin{sageverbatim}
\end{sageverbatim}
%
