%%%%(c)
%%%%(c)  This file is a portion of the source for the textbook
%%%%(c)
%%%%(c)    A First Course in Linear Algebra
%%%%(c)    Copyright 2004 by Robert A. Beezer
%%%%(c)
%%%%(c)  See the file COPYING.txt for copying conditions
%%%%(c)
%%%%(c)
(a)\quad By \acronymref{theorem}{BCS} we can row-reduce $A$, identify pivot columns with the set $D$, and ``keep'' those columns of $A$ and we will have a set with the desired properties.  
%
\begin{align*}
A\rref
\begin{bmatrix}
 \leading{1} & 0 & -13 & -19 \\
 0 & \leading{1} & 8 & 11 \\
 0 & 0 & 0 & 0
\end{bmatrix}
\end{align*}
%
So we have the set of pivot columns $D=\set{1,\,2}$ and we ``keep'' the first two columns of $A$,
%
\begin{align*}
\set{
\colvector{3\\1\\-3},\,
\colvector{5\\2\\-4}
}
\end{align*}
%
(b)\quad We can view the column space as the row space of the transpose (\acronymref{theorem}{CSRST}).  We can get a basis of the row space of a matrix quickly by bringing the matrix to reduced row-echelon form and keeping the nonzero rows as column vectors (\acronymref{theorem}{BRS}).  Here goes.
%
\begin{align*}
\transpose{A}\rref
%
\begin{bmatrix}
 \leading{1} & 0 & -2 \\
 0 & \leading{1} & 3 \\
 0 & 0 & 0 \\
 0 & 0 & 0
\end{bmatrix}
%
\end{align*}
%
Taking the nonzero rows and tilting them up as columns gives us
%
\begin{align*}
\set{
\colvector{1\\0\\-2},\,
\colvector{0\\1\\3}
}
\end{align*}
%
An approach based on the matrix $L$ from extended echelon form (\acronymref{definition}{EEF}) and \acronymref{theorem}{FS} will work as well as an alternative approach.
