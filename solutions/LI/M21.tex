%%%%(c)
%%%%(c)  This file is a portion of the source for the textbook
%%%%(c)
%%%%(c)    A First Course in Linear Algebra
%%%%(c)    Copyright 2004 by Robert A. Beezer
%%%%(c)
%%%%(c)  See the file COPYING.txt for copying conditions
%%%%(c)
%%%%(c)
By \acronymref{definition}{LICV} we can complete this problem by proving that if we assume that
%
\begin{align*}
\alpha_1\left(2\vect{v}_1+3\vect{v}_2+\vect{v}_3\right)+
\alpha_2\left(\vect{v}_1-\vect{v}_2+2\vect{v}_3\right)+
\alpha_3\left(2\vect{v}_1+\vect{v}_2-\vect{v}_3\right)
&=
\zerovector
\end{align*}
%
then we {\em must} conclude that $\alpha_1=\alpha_2=\alpha_3=0$.
Using various properties in \acronymref{theorem}{VSPCV}, we can rearrange this vector equation to
%
\begin{align*}
\left(2\alpha_1+\alpha_2+2\alpha_3\right)\vect{v}_1+
\left(3\alpha_1-\alpha_2+\alpha_3\right)\vect{v}_2+
\left(\alpha_1+2\alpha_2-\alpha_3\right)\vect{v}_3
&=
\zerovector
%
\end{align*}
%
Because the set $S=\set{\vect{v}_1,\,\vect{v}_2,\,\vect{v}_3}$ was assumed to be linearly independent, by \acronymref{definition}{LICV} we {\em must} conclude that
%
\begin{align*}
2\alpha_1+\alpha_2+2\alpha_3&=0\\
3\alpha_1-\alpha_2+\alpha_3&=0\\
\alpha_1+2\alpha_2-\alpha_3&=0
\end{align*}
%
Aah, a homogeneous system of equations.  And it has a unique solution, the trivial solution.  So, $\alpha_1=\alpha_2=\alpha_3=0$, as desired.  It is an inescapable conclusion from our assumption of a relation of linear dependence above.  Done.\par
%
Compare this solution very carefully with \acronymref{solution}{LI.M20}, noting especially how this problem required (and used) the hypothesis that the original set be linearly independent, and how this solution feels more like a proof, while the previous problem could be solved with a fairly simple demonstration of any nontrivial relation of linear dependence.