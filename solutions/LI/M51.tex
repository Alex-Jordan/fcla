%%%%(c)
%%%%(c)  This file is a portion of the source for the textbook
%%%%(c)
%%%%(c)    A First Course in Linear Algebra
%%%%(c)    Copyright 2004 by Robert A. Beezer
%%%%(c)
%%%%(c)  See the file COPYING.txt for copying conditions
%%%%(c)
%%%%(c)
This problem can be solved using the approach in \acronymref{solution}{LI.M50}.  We will provide a solution here that is more ad-hoc, but note that we will have a more straight-forward procedure given by the upcoming \acronymref{theorem}{BS}.\par
%
$\vect{v}_1$ is a non-zero vector, so in a set all by itself we have a linearly independent set.  As $\vect{v}_2$ is a scalar multiple of $\vect{v}_1$, the equation $-4\vect{v}_1+\vect{v}_2=\zerovector$ is a relation of linear dependence on $\set{\vect{v}_1,\,\vect{v}_2}$, so we will pass on $\vect{v}_2$.  No such relation of linear dependence exists on $\set{\vect{v}_1,\,\vect{v}_3}$, though on $\set{\vect{v}_1,\,\vect{v}_3,\,\vect{v}_4}$ we have the relation of linear dependence
$7\vect{v}_1+3\vect{v}_3+\vect{v}_4=\zerovector$.
So take $S=\set{\vect{v}_1,\,\vect{v}_3}$, which is linearly independent.\par
%
Then
%
\begin{align*}
\vect{v}_2&=4\vect{v_1}+0\vect{v_3}
&
\vect{v}_4&=-7\vect{v_1}-3\vect{v_3}
\end{align*}
%
The two equations above are enough to justify the set equality
%
\begin{align*}
W
&=\spn{\set{\vect{v}_1,\,\vect{v}_2,\,\vect{v}_3,\,\vect{v}_4}}
=\spn{\set{\vect{v}_1,\,\vect{v}_3}}
=\spn{S}
\end{align*}
%
There are other solutions (for example, swap the roles of $\vect{v}_1$ and $\vect{v}_2$, but by upcoming theorems we can confidently claim that any solution will be a set $S$ with exactly two vectors.
