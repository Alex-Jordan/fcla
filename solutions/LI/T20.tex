%%%%(c)
%%%%(c)  This file is a portion of the source for the textbook
%%%%(c)
%%%%(c)    A First Course in Linear Algebra
%%%%(c)    Copyright 2004 by Robert A. Beezer
%%%%(c)
%%%%(c)  See the file COPYING.txt for copying conditions
%%%%(c)
%%%%(c)
Our hypothesis and our conclusion use the term linear independence, so it will get a workout.  To establish linear independence, we begin with the definition (\acronymref{definition}{LICV}) and write a relation of linear dependence (\acronymref{definition}{RLDCV}),
%
\begin{equation*}
\alpha_1\left(\vect{v}_1\right)+
\alpha_2\left(\vect{v}_1+\vect{v}_2\right)+
\alpha_3\left(\vect{v}_1+\vect{v}_2+\vect{v}_3\right)+
\alpha_4\left(\vect{v}_1+\vect{v}_2+\vect{v}_3+\vect{v}_4\right)
=\zerovector
\end{equation*}
%
Using the distributive and commutative properties of vector addition and scalar multiplication (\acronymref{theorem}{VSPCV}) this equation can be rearranged as
%
\begin{equation*}
\left(\alpha_1+\alpha_2+\alpha_3+\alpha_4\right)\vect{v}_1+
\left(\alpha_2+\alpha_3+\alpha_4\right)\vect{v}_2+
\left(\alpha_3+\alpha_4\right)\vect{v}_3+
\left(\alpha_4\right)\vect{v}_4
=
\zerovector
\end{equation*}
%
However, this is a relation of linear dependence (\acronymref{definition}{RLDCV}) on a linearly independent set, $\set{\vect{v}_1,\,\vect{v}_2,\,\vect{v}_3,\,\vect{v}_4}$ (this was our lone hypothesis).  By the definition of linear independence (\acronymref{definition}{LICV}) the scalars must all be zero.  This is the homogeneous system of equations,
%
\begin{align*}
\alpha_1+\alpha_2+\alpha_3+\alpha_4&=0\\
\alpha_2+\alpha_3+\alpha_4&=0\\
\alpha_3+\alpha_4&=0\\
\alpha_4&=0
\end{align*}
%
Row-reducing the coefficient matrix of this system (or backsolving) gives the conclusion
%
%
\begin{align*}
\alpha_1=0&&\alpha_2=0&&\alpha_3=0&&\alpha_4=0
\end{align*}
%
This means, by \acronymref{definition}{LICV}, that the original set
%
\begin{equation*}
\set{
\vect{v}_1,\,
\vect{v}_1+\vect{v}_2,\,
\vect{v}_1+\vect{v}_2+\vect{v}_3,\,
\vect{v}_1+\vect{v}_2+\vect{v}_3+\vect{v}_4
}
\end{equation*}
%
is linearly independent.
