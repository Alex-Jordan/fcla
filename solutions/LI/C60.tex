%%%%(c)
%%%%(c)  This file is a portion of the source for the textbook
%%%%(c)
%%%%(c)    A First Course in Linear Algebra
%%%%(c)    Copyright 2004 by Robert A. Beezer
%%%%(c)
%%%%(c)  See the file COPYING.txt for copying conditions
%%%%(c)
%%%%(c)
\acronymref{theorem}{BNS} says that if we find the vector form of the solutions to the homogeneous system $\homosystem{A}$, then the fixed vectors (one per free variable) will have the desired properties.  Row-reduce $A$, viewing it as the augmented matrix of a homogeneous system with an invisible columns of zeros as the last column,
%
\begin{equation*}
\begin{bmatrix}
\leading{1} & 0 & 4 & -5\\
0 & \leading{1} & 2 & -3\\ 
0 & 0 & 0 & 0
\end{bmatrix}
\end{equation*}
%
Moving to the vector form of the solutions (\acronymref{theorem}{VFSLS}), with free variables $x_3$ and $x_4$, solutions to the consistent system (it is homogeneous, \acronymref{theorem}{HSC}) can be expressed as
%
\begin{equation*}
\colvector{x_1\\x_2\\x_3\\x_4}
=
x_3\colvector{-4\\-2\\1\\0}+
x_4\colvector{5\\3\\0\\1}
\end{equation*}
%
Then with $S$ given by 
%
\begin{equation*}
S=\set{\colvector{-4\\-2\\1\\0},\,\colvector{5\\3\\0\\1}}
\end{equation*}
%
\acronymref{theorem}{BNS} guarantees the set has the desired properties.
