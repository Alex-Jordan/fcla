%%%%(c)
%%%%(c)  This file is a portion of the source for the textbook
%%%%(c)
%%%%(c)    A First Course in Linear Algebra
%%%%(c)    Copyright 2004 by Robert A. Beezer
%%%%(c)
%%%%(c)  See the file COPYING.txt for copying conditions
%%%%(c)
%%%%(c)
Form the matrix $C$ whose columns are $\vect{x},\,A\vect{x},\,A^2\vect{x},\,A^3\vect{x},\,A^4\vect{x},\,A^5\vect{x}$ and row-reduce the matrix,
%
\begin{equation*}
\begin{bmatrix}
 0 & 6 & 32 & 102 & 320 & 966 \\
 8 & 10 & 24 & 58 & 168 & 490 \\
 2 & 12 & 50 & 156 & 482 & 1452 \\
 1 & -5 & -47 & -149 & -479 & -1445 \\
 2 & 12 & 50 & 156 & 482 & 1452
\end{bmatrix}
\rref
\begin{bmatrix}
 \leading{1} & 0 & 0 & -3 & -9 & -30 \\
 0 & \leading{1} & 0 & 1 & 0 & 1 \\
 0 & 0 & \leading{1} & 3 & 10 & 30 \\
 0 & 0 & 0 & 0 & 0 & 0 \\
 0 & 0 & 0 & 0 & 0 & 0
\end{bmatrix}
\end{equation*}
%
The simplest possible relation of linear dependence on the columns of $C$ comes from using scalars $\alpha_4=1$ and $\alpha_5=\alpha_6=0$ for the free variables in a solution to $\homosystem{C}$.  The remainder of this solution is $\alpha_1=3$, $\alpha_2=-1$, $\alpha_3=-3$.  This solution gives rise to the polynomial
%
\begin{equation*}
p(x)=3-x-3x^2+x^3=(x-3)(x-1)(x+1)
\end{equation*}
%
which then has the property that $p(A)\vect{x}=\zerovector$.\par
%
No matter how you choose to order the factors of $p(x)$, the value of $k$ (in the language of \acronymref{theorem}{EMHE} and \acronymref{example}{CAEHW}) is $k=2$.  For each of the  three possibilities, we list the resulting eigenvector and the associated eigenvalue:
%
\begin{align*}
(C-3I_5)(C-I_5)\vect{z}&=\colvector{8\\8\\8\\-24\\8}&\lambda&=-1\\
(C-3I_5)(C+I_5)\vect{z}&=\colvector{20\\-20\\20\\-40\\20}&\lambda&=1\\
(C+I_5)(C-I_5)\vect{z}&=\colvector{32\\16\\48\\-48\\48}&\lambda&=3
\end{align*}
%
Note that each of these eigenvectors can be simplified by an appropriate scalar multiple, but we have shown here the actual vector obtained by the product specified in the theorem.
