%%%%(c)
%%%%(c)  This file is a portion of the source for the textbook
%%%%(c)
%%%%(c)    A First Course in Linear Algebra
%%%%(c)    Copyright 2004 by Robert A. Beezer
%%%%(c)
%%%%(c)  See the file COPYING.txt for copying conditions
%%%%(c)
%%%%(c)
These subspaces will be easiest to construct by analyzing a matrix representation of $S$.  Since we can use any matrix representation, we might as well use natural bases that allow us to construct the matrix representation quickly and easily,
%
\begin{align*}
B&=\set{
\begin{bmatrix}1&0\\0&0\end{bmatrix},\,
\begin{bmatrix}0&1\\0&0\end{bmatrix},\,
\begin{bmatrix}0&0\\1&0\end{bmatrix},\,
\begin{bmatrix}0&0\\0&1\end{bmatrix}
}
&
C&=\set{1,\,x,\,x^2}
\end{align*}
%
then we can practically build the matrix representation on sight,
%
\begin{equation*}
\matrixrep{S}{B}{C}=
\begin{bmatrix}
1 & 2 & 5 & -4\\
3 & -1 & 8 & 2\\
1 & 1 & 4 & -2
\end{bmatrix}
\end{equation*}
%
The first step is to find bases for the null space and column space of the matrix representation.  Row-reducing the matrix representation we find,
%
\begin{equation*}
\begin{bmatrix}
\leading{1} & 0 & 3 & 0\\
0 & \leading{1} & 1 & -2\\
0 & 0 & 0 & 0
\end{bmatrix}
\end{equation*}
%
So by \acronymref{theorem}{BNS} and \acronymref{theorem}{BCS}, we have
%
\begin{align*}
\nsp{\matrixrep{S}{B}{C}}&=\spn{\set{\colvector{-3\\-1\\1\\0},\,\colvector{0\\2\\0\\1}}}
&
\csp{\matrixrep{S}{B}{C}}&=\spn{\set{\colvector{1\\3\\1},\,\colvector{2\\-1\\1}}}
\end{align*}
%
Now, the proofs of \acronymref{theorem}{KNSI} and \acronymref{theorem}{RCSI} tell us that we can apply $\ltinverse{\vectrepname{B}}$ and $\vectrepinvname{C}$ (respectively) to ``un-coordinatize'' and get bases for the kernel and range of the linear transformation $S$ itself,
%
\begin{align*}
\krn{S}&=\spn{\set{
\begin{bmatrix}
-3&-1\\1&0
\end{bmatrix},\,
\begin{bmatrix}
0&2\\0&1
\end{bmatrix}
}}
&
\rng{S}&=\spn{\set{1+3x+x^2,\,2-x+x^2}}
\end{align*}
%