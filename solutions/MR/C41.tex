%%%%(c)
%%%%(c)  This file is a portion of the source for the textbook
%%%%(c)
%%%%(c)    A First Course in Linear Algebra
%%%%(c)    Copyright 2004 by Robert A. Beezer
%%%%(c)
%%%%(c)  See the file COPYING.txt for copying conditions
%%%%(c)
%%%%(c)
First, build a matrix representation of $S$ (\acronymref{definition}{MR}).  We are free to choose whatever bases we wish, so we should choose ones that are easy to work with, such as
%
\begin{align*}
B&=\set{1,\,x}\\
C&=\set{\begin{bmatrix}1 & 0\end{bmatrix},\,\begin{bmatrix}0 & 1\end{bmatrix}}
\end{align*}
%
The resulting matrix representation is then
%
\begin{equation*}
\matrixrep{T}{B}{C}=
\begin{bmatrix}
3 & 1\\
2 & 1
\end{bmatrix}
\end{equation*}
%
this matrix is invertible, since it has a nonzero determinant, so by \acronymref{theorem}{IMR} the linear transformation $S$ is invertible.  We can use the matrix inverse and \acronymref{theorem}{IMR} to find a formula for the inverse linear transformation,
%
\begin{align*}
\lt{\ltinverse{S}}{\begin{bmatrix}a&b\end{bmatrix}}
&=\vectrepinv{B}{\matrixrep{\ltinverse{S}}{C}{B}\vectrep{C}{\begin{bmatrix}a&b\end{bmatrix}}}&&\text{\acronymref{theorem}{FTMR}}\\
&=\vectrepinv{B}{\inverse{\left(\matrixrep{S}{B}{C}\right)}\vectrep{C}{\begin{bmatrix}a&b\end{bmatrix}}}&&\text{\acronymref{theorem}{IMR}}\\
&=\vectrepinv{B}{\inverse{\left(\matrixrep{S}{B}{C}\right)}\colvector{a\\b}}&&\text{\acronymref{definition}{VR}}\\
&=\vectrepinv{B}{
\inverse{\left(
\begin{bmatrix}
3 & 1\\
2 & 1
\end{bmatrix}
\right)}
\colvector{a\\b}}\\
&=\vectrepinv{B}{
\begin{bmatrix}
1 & -1\\
-2 & 3
\end{bmatrix}
\colvector{a\\b}}&&\text{\acronymref{definition}{MI}}\\
&=\vectrepinv{B}{
\colvector{
a-b\\
-2a+3b
}}&&\text{\acronymref{definition}{MVP}}\\
&=(a-b)+(-2a+3b)x&&\text{\acronymref{definition}{VR}}
\end{align*}