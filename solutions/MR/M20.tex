%%%%(c)
%%%%(c)  This file is a portion of the source for the textbook
%%%%(c)
%%%%(c)    A First Course in Linear Algebra
%%%%(c)    Copyright 2004 by Robert A. Beezer
%%%%(c)
%%%%(c)  See the file COPYING.txt for copying conditions
%%%%(c)
%%%%(c)
Build a matrix representation (\acronymref{definition}{MR}) with the set
%
\begin{equation*}
B=\set{1,\,x,\,x^2,\,\dots,\,x^n}
\end{equation*}
%
employed as a basis of both the domain and codomain.  Then
%
\begin{align*}
\vectrep{B}{\lt{D}{1}}&=\vectrep{B}{0}
=\colvector{0\\0\\0\\\vdots\\0\\0}
&
\vectrep{B}{\lt{D}{x}}&=\vectrep{B}{1}
=\colvector{1\\0\\0\\\vdots\\0\\0}\\
%
\vectrep{B}{\lt{D}{x^2}}&=\vectrep{B}{2x}
=\colvector{0\\2\\0\\\vdots\\0\\0}
&
\vectrep{B}{\lt{D}{x^3}}&=\vectrep{B}{3x^2}
=\colvector{0\\0\\3\\\vdots\\0\\0}\\
%
&\vdots\\
%
\vectrep{B}{\lt{D}{x^n}}&=\vectrep{B}{nx^{n-1}}
=\colvector{0\\0\\0\\\vdots\\n\\0}\\
%
\end{align*}
%
and the resulting matrix representation is
%
\begin{equation*}
\matrixrep{D}{B}{B}=
\begin{bmatrix}
0 & 1 & 0 & 0 & \dots & 0 & 0 \\
0 & 0 & 2 & 0 & \dots & 0 & 0 \\
0 & 0 & 0 & 3 & \dots & 0 & 0 \\
   & \vdots &&& \ddots & & \vdots \\
0 & 0 & 0 & 0 & \dots & 0 & n \\
0 & 0 & 0 & 0 & \dots & 0 & 0 
\end{bmatrix}
\end{equation*}
%
This $(n+1)\times(n+1)$ matrix is very close to being in reduced row-echelon form.  Multiply row $i$ by $\frac{1}{i}$, for $1\leq i\leq n$, to convert it to reduced row-echelon form.  From this we can see that matrix representation $\matrixrep{D}{B}{B}$ has rank $n$ and nullity $1$.  Applying \acronymref{theorem}{RCSI} and \acronymref{theorem}{KNSI} tells us that the linear transformation $D$ will have the same values for the rank and nullity, as well.