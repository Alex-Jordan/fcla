%%%%(c)
%%%%(c)  This file is a portion of the source for the textbook
%%%%(c)
%%%%(c)    A First Course in Linear Algebra
%%%%(c)    Copyright 2004 by Robert A. Beezer
%%%%(c)
%%%%(c)  See the file COPYING.txt for copying conditions
%%%%(c)
%%%%(c)
By \acronymref{definition}{SIM} we know that there is a nonsingular matrix $S$ so that $A=\similar{B}{S}$.  Then
%
\begin{align*}
A^3&=(\similar{B}{S})^3\\
&=(\similar{B}{S})(\similar{B}{S})(\similar{B}{S})\\
&=\inverse{S}B(S\inverse{S})B(S\inverse{S})BS&&\text{\acronymref{theorem}{MMA}}\\
&=\inverse{S}B(I_3)B(I_3)BS&&\text{\acronymref{definition}{MI}}\\
&=\inverse{S}BBBS&&\text{\acronymref{theorem}{MMIM}}\\
&=\inverse{S}B^3S
\end{align*}
%
This equation says that $A^3$ is similar to $B^3$ (via the matrix $S$).\par
%
More generally, if $A$ is similar to $B$, and $m$ is a non-negative integer, then $A^m$ is similar to $B^m$.  This can be proved using induction (\acronymref{technique}{I}).