%%%%(c)
%%%%(c)  This file is a portion of the source for the textbook
%%%%(c)
%%%%(c)    A First Course in Linear Algebra
%%%%(c)    Copyright 2004 by Robert A. Beezer
%%%%(c)
%%%%(c)  See the file COPYING.txt for copying conditions
%%%%(c)
%%%%(c)
$\linearsystem{A}{\vect{b}}$ must be homogeneous.  To see this consider that 
%
\begin{align*}
\vect{b}
&=A\vect{x}&&\text{\acronymref{theorem}{SLEMM}}\\
&=A\vect{x}+\zerovector&&\text{\acronymref{property}{ZC}}\\
&=A\vect{x}+A\vect{y}-A\vect{y}&&\text{\acronymref{property}{AIC}}\\
&=A\left(\vect{x}+\vect{y}\right)-A\vect{y}&&\text{\acronymref{theorem}{MMDAA}}\\
&=\vect{b}-\vect{b}&&\text{\acronymref{theorem}{SLEMM}}\\
&=\zerovector&&\text{\acronymref{property}{AIC}}
\end{align*}
%
By \acronymref{definition}{HS} we see that $\linearsystem{A}{\vect{b}}$ is homogeneous.
