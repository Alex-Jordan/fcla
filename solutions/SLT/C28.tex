%%%%(c)
%%%%(c)  This file is a portion of the source for the textbook
%%%%(c)
%%%%(c)    A First Course in Linear Algebra
%%%%(c)    Copyright 2004 by Robert A. Beezer
%%%%(c)
%%%%(c)  See the file COPYING.txt for copying conditions
%%%%(c)
%%%%(c)
The range of $T$ is 
\begin{align*}
\rng{T} 
&= \setparts{
\begin{bmatrix} 
a + b & a + b + c\\ 
a + b + c& a + d
\end{bmatrix}}{a,b,c,d\in\complexes}\\
&= \setparts{
a\begin{bmatrix} 1 & 1 \\1 & 1 \end{bmatrix} + 
b\begin{bmatrix} 1 & 1 \\1 & 0 \end{bmatrix} +
c\begin{bmatrix} 0 & 1 \\1 & 0 \end{bmatrix} +
d\begin{bmatrix} 0 & 0 \\0 & 1 \end{bmatrix}
}{a,b,c,d\in\complexes}\\
&=\spn{
\begin{bmatrix} 1 & 1 \\1 & 1 \end{bmatrix},
\begin{bmatrix} 1 & 1 \\1 & 0 \end{bmatrix}, 
\begin{bmatrix} 0 & 1 \\1 & 0 \end{bmatrix}, 
\begin{bmatrix} 0 & 0 \\0 & 1 \end{bmatrix}}\\
&=\spn{
\begin{bmatrix} 1 & 1 \\1 & 0 \end{bmatrix}, 
\begin{bmatrix} 0 & 1 \\1 & 0 \end{bmatrix}, 
\begin{bmatrix} 0 & 0 \\0 & 1 \end{bmatrix}}.
\end{align*}
%
Can you explain the last equality above?\par
%
These three matrices are linearly independent, so a basis of $\rng{T}$ is 
$\set{
\begin{bmatrix} 1 & 1 \\1 & 0 \end{bmatrix}, 
\begin{bmatrix} 0 & 1 \\1 & 0 \end{bmatrix}, 
\begin{bmatrix} 0 & 0 \\0 & 1 \end{bmatrix}}$.  Thus, $T$ is not surjective, since the range has dimension 3 which is shy of $\dimension{M_{2,2}}=4$.
(Notice that the range is actually the subspace of symmetric $2 \times 2$ matrices in $M_{2,2}$.)