%%%%(c)
%%%%(c)  This file is a portion of the source for the textbook
%%%%(c)
%%%%(c)    A First Course in Linear Algebra
%%%%(c)    Copyright 2004 by Robert A. Beezer
%%%%(c)
%%%%(c)  See the file COPYING.txt for copying conditions
%%%%(c)
%%%%(c)
Let $a$ be the hundreds digit, $b$ the tens digit, and $c$ the ones digit.  Then the first condition says that $b+c=5$.  The original number is $100a+10b+c$, while the reversed number is $100c+10b+a$.  So the second condition is
%
\begin{equation*}
792=\left(100a+10b+c\right)-\left(100c+10b+a\right)=99a-99c
\end{equation*}
%
So we arrive at the system of equations
%
\begin{align*}
b+c&=5\\
99a-99c&=792
\end{align*}
%
Using equation operations, we arrive at the equivalent system
%
\begin{align*}
a-c&=8\\
b+c&=5
\end{align*}
%
We can vary $c$ and obtain infinitely many solutions.  However, $c$ must be a digit, restricting us to ten values (0 -- 9).  Furthermore, if $c>1$, then the first equation forces $a>9$, an impossibility.    Setting $c=0$, yields $850$ as a solution, and setting $c=1$ yields $941$ as another solution.
