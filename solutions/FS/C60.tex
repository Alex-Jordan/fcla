%%%%(c)
%%%%(c)  This file is a portion of the source for the textbook
%%%%(c)
%%%%(c)    A First Course in Linear Algebra
%%%%(c)    Copyright 2004 by Robert A. Beezer
%%%%(c)
%%%%(c)  See the file COPYING.txt for copying conditions
%%%%(c)
%%%%(c)
(a)\quad The definition of the column space is the span of the set of columns (\acronymref{definition}{CSM}).  So the desired set is just the four columns of $B$,
%
\begin{equation*}
S=\set{
\colvector{2\\1\\-1},\,
\colvector{3\\1\\2},\,
\colvector{1\\0\\3},\,
\colvector{1\\1\\-4}
}
\end{equation*}
%
%
(b)\quad  
\acronymref{theorem}{BCS} suggests row-reducing the matrix and using the columns of $B$ that correspond to the pivot columns.  
%
\begin{equation*}
B\rref
\begin{bmatrix}
\leading{1} & 0 & -1 & 2\\
0 & \leading{1} & 1 & -1\\
0 & 0 & 0 & 0
\end{bmatrix}
\end{equation*}
%
So the pivot columns are numbered by elements of $D=\set{1,\,2}$, so the requested set is
%
\begin{equation*}
S=\set{
\colvector{2\\1\\-1},\,
\colvector{3\\1\\2}}
\end{equation*}
%
(c)\quad 
We can find this set by row-reducing the transpose of $B$, deleting the zero rows, and using the nonzero rows as column vectors in the set.  This is an application of \acronymref{theorem}{CSRST} followed by \acronymref{theorem}{BRS}.
%
\begin{equation*}
\transpose{B}\rref
\begin{bmatrix}
\leading{1} & 0 & 3\\
0 & \leading{1} & -7\\
0 & 0 & 0\\
0 & 0 & 0
\end{bmatrix}
\end{equation*}
%
So the requested set is
%
\begin{equation*}
S=\set{
\colvector{1\\0\\3},\,
\colvector{0\\1\\-7}
}
\end{equation*}
%
(d)\quad 
With the column space expressed as a null space, the vectors obtained via \acronymref{theorem}{BNS} will be of the desired shape.  So we first proceed with \acronymref{theorem}{FS} and create the extended echelon form,
%
\begin{equation*}
\augmented{B}{I_3}=
\begin{bmatrix}
2 & 3 & 1 & 1 & 1 & 0 & 0\\
1 & 1 & 0 & 1 & 0 & 1 & 0\\
-1 & 2 & 3 & -4 & 0 & 0 & 1
\end{bmatrix}
\rref
\begin{bmatrix}
\leading{1} & 0 & -1 & 2 & 0 & \frac{2}{3} & \frac{-1}{3}\\
0 & \leading{1} & 1 & -1 & 0 & \frac{1}{3} & \frac{1}{3}\\
0 & 0 & 0 & 0 & \leading{1} & \frac{-7}{3} & \frac{-1}{3}
\end{bmatrix}
\end{equation*}
So, employing \acronymref{theorem}{FS}, we have $\csp{B}=\nsp{L}$, where 
%
\begin{equation*}
L=
\begin{bmatrix}
\leading{1} & \frac{-7}{3} & \frac{-1}{3}
\end{bmatrix}
\end{equation*}
%
We can find the desired set of vectors from \acronymref{theorem}{BNS} as
%
\begin{equation*}
S=\set{
\colvector{\frac{7}{3}\\1\\0},\,
\colvector{\frac{1}{3}\\0\\1}
}
\end{equation*}
%