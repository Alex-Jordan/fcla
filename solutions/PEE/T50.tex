%%%%(c)
%%%%(c)  This file is a portion of the source for the textbook
%%%%(c)
%%%%(c)    A First Course in Linear Algebra
%%%%(c)    Copyright 2004 by Robert A. Beezer
%%%%(c)
%%%%(c)  See the file COPYING.txt for copying conditions
%%%%(c)
%%%%(c)
Since $\lambda$ is an eigenvalue of a nonsingular matrix, $\lambda\neq 0$ (\acronymref{theorem}{SMZE}).  $A$ is invertible (\acronymref{theorem}{NI}), and so $-\lambda A$ is invertible (\acronymref{theorem}{MISM}). Thus $-\lambda A$ is nonsingular (\acronymref{theorem}{NI}) and $\detname{-\lambda A}\neq 0$ (\acronymref{theorem}{SMZD}).   
%
\begin{align*}
\charpoly{\inverse{A}}{\frac{1}{\lambda}}
%
&=\detname{\inverse{A}-\frac{1}{\lambda}I_n}
&&\text{\acronymref{definition}{CP}}\\
%
&=1\detname{\inverse{A}-\frac{1}{\lambda}I_n}
&&\text{\acronymref{property}{OCN}}\\
%
&=\frac{1}{\detname{-\lambda A}}
\detname{-\lambda A}\detname{\inverse{A}-\frac{1}{\lambda}I_n}
&&\text{\acronymref{property}{MICN}}\\
%
&=\frac{1}{\detname{-\lambda A}}
\detname{\left(-\lambda A\right)\left(\inverse{A}-\frac{1}{\lambda}I_n\right)}
&&\text{\acronymref{theorem}{DRMM}}\\
%
&=\frac{1}{\detname{-\lambda A}}
\detname{-\lambda A\inverse{A}-\left(-\lambda A\right)\frac{1}{\lambda}I_n}
&&\text{\acronymref{theorem}{MMDAA}}\\
%
&=\frac{1}{\detname{-\lambda A}}
\detname{-\lambda I_n-\left(-\lambda A\right)\frac{1}{\lambda}I_n}
&&\text{\acronymref{definition}{MI}}\\
%
&=\frac{1}{\detname{-\lambda A}}
\detname{-\lambda I_n+\lambda\frac{1}{\lambda}AI_n}
&&\text{\acronymref{theorem}{MMSMM}}\\
%
&=\frac{1}{\detname{-\lambda A}}
\detname{-\lambda I_n+1AI_n}
&&\text{\acronymref{property}{MICN}}\\
%
&=\frac{1}{\detname{-\lambda A}}
\detname{-\lambda I_n+AI_n}
&&\text{\acronymref{property}{OCN}}\\
%
&=\frac{1}{\detname{-\lambda A}}
\detname{-\lambda I_n+A}
&&\text{\acronymref{theorem}{MMIM}}\\
%
&=\frac{1}{\detname{-\lambda A}}
\detname{A-\lambda I_n}
&&\text{\acronymref{property}{ACM}}\\
%
&=\frac{1}{\detname{-\lambda A}}
\charpoly{A}{\lambda}&&\text{\acronymref{definition}{CP}}\\
%
&=\frac{1}{\detname{-\lambda A}}\,0&&\text{\acronymref{theorem}{EMRCP}}\\
%
&=0&&\text{\acronymref{property}{ZCN}}
%
\end{align*}
%
So $\frac{1}{\lambda}$ is a root of the characteristic polynomial of $\inverse{A}$ and so is an eigenvalue of $\inverse{A}$.  This proof is due to \sarabucht.