%%%%(c)
%%%%(c)  This file is a portion of the source for the textbook
%%%%(c)
%%%%(c)    A First Course in Linear Algebra
%%%%(c)    Copyright 2004 by Robert A. Beezer
%%%%(c)
%%%%(c)  See the file COPYING.txt for copying conditions
%%%%(c)
%%%%(c)
An arbitrary linear combination is
%
\begin{equation*}
\vect{y}=
3\colvector{2\\-3\\1}+
(-2)\colvector{1\\4\\1}+
1\colvector{7\\-5\\4}+
(-2)\colvector{-7\\-6\\-5}
=
\colvector{25\\-10\\15}
\end{equation*}
%
(You probably used a different collection of scalars.)  We want to write $\vect{y}$ as a linear combination of 
%
\begin{equation*}
B=\set{\colvector{1\\0\\\frac{7}{11}},\,\colvector{0\\1\\\frac{1}{11}}}
\end{equation*}
%
We could set this up as vector equation with variables as scalars in a linear combination of the vectors in $B$, but since the first two slots of $B$ have such a nice pattern of zeros and ones, we can determine the necessary scalars easily and then double-check our answer with a computation in the third slot,
%
\begin{equation*}
25\colvector{1\\0\\\frac{7}{11}}+(-10)\colvector{0\\1\\\frac{1}{11}}
=
\colvector{25\\-10\\(25)\frac{7}{11}+(-10)\frac{1}{11}}
=
\colvector{25\\-10\\15}=\vect{y}
\end{equation*}
%
Notice how the uniqueness of these scalars arises.  They are {\em forced} to be $25$ and $-10$.



