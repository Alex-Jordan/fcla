%%%%(c)
%%%%(c)  This file is a portion of the source for the textbook
%%%%(c)
%%%%(c)    A First Course in Linear Algebra
%%%%(c)    Copyright 2004 by Robert A. Beezer
%%%%(c)
%%%%(c)  See the file COPYING.txt for copying conditions
%%%%(c)
%%%%(c)
\begin{enumerate}
\item 
The span of a single vector $\vect{v}$ is the set of all linear combinations of that vector.  Thus, $\spn{\vect{v}} = \setparts{\alpha\vect{v}}{\alpha\in\real{\relax}}$.  This is the line through the origin and containing the (geometric) vector $\vect{v}$.  Thus, if $\vect{v} = \colvector{v_1\\v_2\\v_3}$, then the span of $\vect{v}$ is the line through $(0,0,0)$ and $(v_1, v_2, v_3)$.
%
\item 
If the two vectors point in the same direction, then their span is the line through them.  Recall that while two points determine a line, three points determine a plane. Two vectors will span a plane if they point in different directions, meaning that $\vect{u}$ does not lie on the line through $\vect{v}$ and vice-versa.  The plane spanned by $\vect{u} = \colvector{u_1\\u_1\\u_1}$
and  $\vect{v} = \colvector{v_1\\v_2\\v_3}$  is determined by the origin and the points $(u_1, u_2, u_3)$ and $(v_1, v_2, v_3)$. 
%
\item 
If all three vectors lie on the same line, then the span is that line.  If one is a linear combination of the other two, but they are not all on the same line, then they will lie in a plane.  Otherwise, the span of the set of three vectors will be all of 3-space.
\end{enumerate}