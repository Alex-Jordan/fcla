%%%%(c)
%%%%(c)  This file is a portion of the source for the textbook
%%%%(c)
%%%%(c)    A First Course in Linear Algebra
%%%%(c)    Copyright 2004 by Robert A. Beezer
%%%%(c)
%%%%(c)  See the file COPYING.txt for copying conditions
%%%%(c)
%%%%(c)
(a)\quad
\acronymref{theorem}{SSNS} provides formulas for a set $S$ with this property, but first we must row-reduce $A$
%
\begin{equation*}
A\rref
\begin{bmatrix}
 \leading{1} & 0 & -1 & -1 \\
 0 & \leading{1} & 1 & 2 \\
 0 & 0 & 0 & 0
\end{bmatrix}
\end{equation*}
%
$x_3$ and $x_4$ would be the free variables in the homogeneous system $\homosystem{A}$ and \acronymref{theorem}{SSNS} provides the set $S=\set{\vect{z}_1,\,\vect{z}_2}$ where
%
\begin{align*}
\vect{z}_1
&=
\colvector{1\\-1\\1\\0}
&
\vect{z}_2
&=
\colvector{1\\-2\\0\\1}
\end{align*}
%
(b)\quad
Simply employ the components of the vector $\vect{z}$ as the variables in the homogeneous system $\homosystem{A}$.  The three equations of this system evaluate as follows,
%
\begin{align*}
 2(3) + 3(-5)+ 1(1) + 4(2)&=0 \\
 1(3) + 2(-5)+ 1(1) + 3(2)&= 0\\
 -1(3) + 0(-5)+ 1(1) + 1(2)&=0
\end{align*}
%
Since each result is zero, $\vect{z}$ qualifies for membership in $\nsp{A}$.\par
%
(c)\quad 
By \acronymref{theorem}{SSNS} we know this must be possible (that is the moral of this exercise).  Find scalars $\alpha_1$ and $\alpha_2$ so that
%
\begin{equation*}
\alpha_1\vect{z}_1+\alpha_2\vect{z}_2
=
\alpha_1\colvector{1\\-1\\1\\0}+\alpha_2\colvector{1\\-2\\0\\1}
=
\colvector{3 \\ -5 \\ 1 \\ 2}
=
\vect{z}
\end{equation*}
%
\acronymref{theorem}{SLSLC} allows us to convert this question into a question about a system of four equations in two variables.  The augmented matrix of this system row-reduces to 
%
\begin{equation*}
\begin{bmatrix}
 \leading{1} & 0 & 1 \\
 0 & \leading{1} & 2 \\
 0 & 0 & 0 \\
 0 & 0 & 0
\end{bmatrix}
\end{equation*}
%
A solution is $\alpha_1=1$ and $\alpha_2=2$.  (Notice too that this solution is unique!)
