%%%%(c)
%%%%(c)  This file is a portion of the source for the textbook
%%%%(c)
%%%%(c)    A First Course in Linear Algebra
%%%%(c)    Copyright 2004 by Robert A. Beezer
%%%%(c)
%%%%(c)  See the file COPYING.txt for copying conditions
%%%%(c)
%%%%(c)
\begin{enumerate}
\item
The span of a single vector $\vect{v}$ is the set of all linear combinations of that vector.  Thus, 
$\spn{\vect{v}} = \setparts{\alpha\vect{v}}{\alpha\in\real{\relax}}$.  
This is the line through the origin and containing the (geometric) vector $\vect{v}$.  Thus, if 
$\vect{v}=\colvector{v_1\\v_2}$, 
then the span of $\vect{v}$ is the line through $(0,0)$ and $(v_1,v_2)$.
%
\item 
Two vectors will span the entire plane if they point in different directions, meaning that $\vect{u}$ does not lie on the line through $\vect{v}$ and vice-versa.  That is, for vectors $\vect{u}$ and $\vect{v}$ in $\real{2}$, $\spn{\vect{u}, \vect{v}} = \real{2}$ if $\vect{u}$ is not a multiple of $\vect{v}$.
\end{enumerate}