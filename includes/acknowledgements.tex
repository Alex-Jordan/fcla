%%%%(c)
%%%%(c)  This file is a portion of the source for the textbook
%%%%(c)
%%%%(c)    A First Course in Linear Algebra
%%%%(c)    Copyright 2004 by Robert A. Beezer
%%%%(c)
%%%%(c)  See the file COPYING.txt for copying conditions
%%%%(c)
%%%%(c)
Many people have helped to make this book, and its freedoms, possible.\par
%
First, the time to create, edit and distribute the book has been provided implicitly and explicitly by the University of Puget Sound.  A sabbatical leave Spring 2004 and a course release in Spring 2007 are two obvious examples of explicit support.  The latter was provided by support from the Lind-VanEnkevort Fund.  The university has also provided clerical support, computer hardware, network servers and bandwidth.  Thanks to Dean Kris Bartanen and the chair of the Mathematics and Computer Science Department, Professor Martin Jackson, for their support, encouragement and flexibility.\par
%
My colleagues in the Mathematics and Computer Science Department have graciously taught our introductory linear algebra course using preliminary versions and have provided valuable suggestions that have improved the book immeasurably.  Thanks to Professor Martin Jackson (v0.30), Professor David Scott (v0.70) and Professor Bryan Smith (v0.70, 0.80, v1.00).\par
%
University of Puget Sound librarians Lori Ricigliano, Elizabeth Knight and Jeanne Kimura provided valuable advice on production, and interesting conversations about copyrights.\par
%
Many aspects of the book have been influenced by insightful questions and creative suggestions from the students who have labored through the book in our courses.  For example, the flashcards with theorems and definitions are a direct result of a student suggestion.  I will single out a handful of students have been especially adept at finding and reporting mathematically significant typographical errors: 
% version 0.10
Jake Linenthal, Christie Su, Kim Le, Sarah McQuate, 
% version 0.30
Andy Zimmer, 
% version 0.50
Travis Osborne, Andrew Tapay, Mark Shoemaker, Tasha Underhill, 
% version 0.70
Tim Zitzer,
% version 1.00
Elizabeth Million,
% version 1.20
% None
% version 1.30
and Steve Canfield.
%
\par
%
I have tried to be as original as possible in the organization and presentation of this beautiful subject.  However, I have been influenced by many years of teaching from another excellent textbook, {\sl Introduction to Linear Algebra} by L.W.\ Johnson, R.D.\ Reiss and J.T.\ Arnold.  When I have needed inspiration for the correct approach to particularly important proofs, I have learned to eventually consult two other textbooks.  Sheldon Axler's {\sl Linear Algebra Done Right} is a highly original exposition, while Ben Noble's {\sl Applied Linear Algebra} frequently strikes just the right note between rigor and intuition.  Noble's excellent book is highly recommended, even though its publication dates to 1969.\par
%
Conversion to various electronic formats have greatly depended on assistance from:  Eitan Gurari, author of the powerful \LaTeX\ translator, {\tt tex4ht}; Davide Cervone, author of {\tt jsMath}; and Carl Witty, who advised and tested the Sony Reader format.  Thanks to these individuals for their critical assistance.\par
%
General support and encouragement of free and affordable textbooks, in addition to specific promotion of this text, was provided by Nicole Allen, Textbook Advocate at Student Public Interest Research Groups.  Nicole was an early consumer of this material, back when it looked more like lecture notes than a textbook.\par
%
Finally, in every possible case, the production and distribution of this book has been accomplished with open-source software.  The range of individuals and projects is far too great to pretend to list them all.  The book's web site will someday maintain pointers to as many of these projects as possible.