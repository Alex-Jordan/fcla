%%%%(c)
%%%%(c)  This file is a portion of the source for the textbook
%%%%(c)
%%%%(c)    A First Course in Linear Algebra
%%%%(c)    Copyright 2004 by Robert A. Beezer
%%%%(c)
%%%%(c)  See the file COPYING.txt for copying conditions
%%%%(c)
%%%%(c)
This appendix contains important ideas about complex numbers, sets, and the logic and techniques of forming proofs.  It is not meant to be read straight through, but you should head here when you need to review these ideas.\par
%
We choose to expand the set of scalars from the real numbers, $\real{\null}$, to the set of complex numbers, $\complex{\null}$.  So basic operations with complex numbers (like addition and division) will be necessary.  This can be safely postponed until your arrival in \acronymref{section}{O}, and a refresher before \acronymref{chapter}{E} would be a good idea as well.\par
%
Sets are extremely important in all of mathematics, but maybe you have not had much exposure to the basic operations.  Check out \acronymref{section}{SET}.  The text will send you here frequently as well.  Visit often.\par
%
\ifthenelse{\boolean{techniquesappendix}}{
This book is as much about {\em doing} mathematics as it is about linear algebra.  The ``Proof Techniques'' are vignettes about logic, types of theorems, structure of proofs, or just plain old-fashioned advice about how to {\em do} advanced mathematics.  The text will frequently point to one of these techniques in advance of their first use, and for specific instructions there will be additional references.  If you find constructing proofs difficult (we all did once), then head back here and browse through the advice for second or third readings.
}{\relax}

